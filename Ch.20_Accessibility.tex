\chapter{Web accessibility\label{accessibility}}

Web accessibility aims to ensure an equitable online experience for users of diverse abilities. According to World Health Organization, an estimate of over 1.3 billion people worldwide face some form of significant disability \citep{whodisability}. Persons with low vision, hard of hearing or learning disability might have difficulties perceiving information on the web. The significance of web accessibility is increasingly apparent. This chapter delve into web accessibility. In section 2.1....

\section{Definition}

Web accessibility is defined as an equal opportunity to perceive, operate, understand and interact with an online service \citep{webaccessibilitydefinition}. Regardless of disability a person faces, the same information and functionality should be made available. Web accessibility considers the technical and content implementation. Technical implementation of web accessibility ensures that barriers for assistive technologies, such as screen readers or speech input software, are removed. Structuring content with headings, using short paragraphs and simple understandable language makes the content accessible. Additionally, content accessibility is providing content in multiple forms, such as videos, sound, images and text.

\section{Disabilities affecting the use of web}

Spanning from visual impairments to situational disabilities, each condition presents barriers that hinder the ability to interact and comprehend information on the web. Understanding these disabilities is crucial when designing online services that prioritize inclusion and ensures an equal experience for all users. Disabilities that affect using the web are vision, physical, auditory and cognitive \citep{w3cbarriers}.

One of the most commonly known condition affecting web accessibility is vision impairment and blindness with more then 2.2 billion people worldwide impacted \citep[Chapter~1]{whovision, webaccessibility}. Individuals with visual impairments depend on functionality for formatting the content to their needs. Barriers encountered by individuals with visual impairment are lack of text alternatives for images and controls, insufficient contrast between text color and background color and lack of keyboard navigation support. 

Accessibility is not only targeted for individuals with impairments. Individuals that have lost their glasses, has a broken arm or a hard time reading the screen when the sun glares also benefits of web accessibility. In addition to temporal disabilities and situational limitations, also aging people benefit. Most of the people with low vision are above the age of 50 \citep{whovision}.


\section{Implementation and standards}

Web accessibility strives for an equal opportunity for persons with any sort of impairment to interact with web services on any device. In web accessibility the goal is to implement a web page in a specific way so that assistive technologies can help to remove the barrier for accessing the information provided. Commonly known assistive technologies are screen readers, screen magnifiers and speech input software that are incorporated in major operating systems.


Web designers, developers and content creators implement web accessibility. Accessibility is built upon these three topics, technical implementation, ease of use and comprehensibility of content \citep{webaccessibilitydefinition}. Technical implementation provides functionality for assistive technologies in the web source code based on web standards. Ease of use involves having a user experience with clearly structured frame for navigation and content placement. Comprehensibility of content is the structure of the content itself, meaning the content is written in short paragraphs and plain language. 

In web development the technical implementation is mainly done by web developers by following the HTML and CSS standard, ease of use is web designers responsibility and page content itself is written and structured by content creators. However, there are overlap in responsibilities of implementing accessibility, as creation of web pages has become easier due to website builders and different web frameworks. In the EU, the implementation details for web accessibility have been standardized based on existing guidelines.



\section{Legislation}

The emergence of directives and legislative measures plays a role in the goal to develop best practices for web accessibility. The European Union (EU) Directive 2016/2102 was adopted in 2016 to ratify part of the United Nations Convention on the Rights of Persons with Disabilities (UN CRPD). The Directive enforces EU Member States to ensure that their public sector organizations implement and monitor accessibility on their web pages and mobile applications \citep{eudirective2016}. The directive gave a clear timeline for the member states. At latest of September 2018 each member state should have transposed this directive into national legislation. Since September 2019 all new web sites must conform. All public sector websites had to comply after September 2020. In June 2021 mobile applications had to conform and in December 2021 the monitoring and reporting started. 

The EU Directive adopted in 2016 refers to the European Standard EN 301 549 that specifies the accessibility requirements for ICT products and services. In the first version of the Standard the guidelines and requirements were based on Web Content Accessibility Guidelines provided by the World Wide Web Consortium \citep{wcagadoptioneurope}. During the same time as member states had to transpose this into a national law in 2018, the standard was changed to be a direct reference to the WCAG making the WCAG an accessibility standard that EU member states should follow. There are three levels of web accessibility defined: A being the lowest, AA and AAA being the highest. The directive requires the public sector to conform with AA level of the guidelines.

Monitoring of accessibility in the public sector should be done every three years \citep{eudirectivemonitoring}. The first monitoring and accessibility report from member states to the EU commission was given in 2021. Next reports should be delivered in December 2024. This monitoring is divided into two review categories, in-depth and simplified. Simplified review requires accessibility testing using only automated accessibility evaluation tools, where as in-depth review requires both tools and manual review by accessibility specialists. 

Based on the public sector directive, the EU Commission has created a similar directive affecting some private sector bodies. The European Accessibility Act (EAA) was adopted by the EU in 2019 to complement the 2016 directive and UN CRPD \citep{eudirective2019}. The EAA will require medium and large sized private sector companies in the field of banking, travel and e-commerce starting to follow accessibility guidelines from June 2025 forward.

\section{Guidelines}

The World Wide Web Consortium creates accessibility guidelines through the Web Accessibility Initiative \citep{wcagoverview}. The international standard is named Web Accessibility Content Guidelines (WCAG). Version 2 of WCAG was published in 2008. Version 2.1 came out in 2018 and is the version referenced in the European Standard. Version 2.2 was published, and made a recommendation by W3C in October 2023. The W3C recommends always to use the latest version as they are backwards compatible. The EU commission is in progress of harmonising the EN 301 549 Standard with the latest WCAG version with a planned release of 2025 \citep{etsi_standard}.

The guidelines aim to ensure that digital content is accessible to a broad audience and adaptable to various forms, accommodating diverse sensory, physical, and cognitive abilities of individuals \citep{wcag22}. The WCAG document is written in a technology agnostic way. It is based on four principles: \textit{perceivable, operable, understandable, and robust}. In each principle, there are guidelines and under each guideline are success criteria. There are a total of 13 guidelines and 86 success criteria. In total there are three levels of accessibility; A being the lowest, AA, and AAA being the highest, see table \ref{table:wcag22}. 

Each success criterion is part of one level of accessibility. Version 2.2 remove one and add nine new success criteria. Success criterion 4.1.1 is made obsolete, as assistive technologies do not need to directly parse the HTML code anymore, and problems occurred by this criterion is addressed by other criteria. The new nine added success criteria are: 2.4.11, 2.4.12, 2.4.13, 2.5.7, 2.5.8, 3.2.6, 3.3.7, 3.3.8 and 3.3.9 as shown in table \ref{table:wcag22}. Six of these are required to be implemented to achieve AA-level of accessibility as per legal requirements.

\begin{table}[]
\caption{WCAG 2.2 principles, guidelines and success criteria divided with the three accessibility levels. Adapted from \textcite{wcag22}}
\begin{adjustbox}{width=1\textwidth}
\begin{tabular}{|l|l|l|l|l|}
\hline
\textbf{Principle} &
  \multicolumn{1}{c|}{\textbf{Guideline}} &
  \multicolumn{1}{c|}{\textbf{\begin{tabular}[c]{@{}c@{}}Success Criterion\\ Level A\end{tabular}}} &
  \multicolumn{1}{c|}{\textbf{\begin{tabular}[c]{@{}c@{}}Success Criterion\\ Level AA\end{tabular}}} &
  \multicolumn{1}{c|}{\textbf{\begin{tabular}[c]{@{}c@{}}Success Criterion\\ Level AAA\end{tabular}}} \\ \hline
\multirow{4}{*}{\textbf{Perceivable}} &
  1.1 Text alternatives &
  1.1.1 Non-text content &
   &
   \\ \cline{2-5} 
 &
  \begin{tabular}[c]{@{}l@{}}1.2 Time-based media \\ (Prerecorded, \\ unless stated otherwise)  \end{tabular} &
  \begin{tabular}[c]{@{}l@{}}1.2.1 Audio-only and \\ Video-only\\ 1.2.2 Captions\\ 1.2.3 Audio description \\ or media alternative\end{tabular} &
  \begin{tabular}[c]{@{}l@{}}1.2.4 Captions (Live)\\ 1.2.5 Audio description\end{tabular} &
  \begin{tabular}[c]{@{}l@{}}1.2.6 Sign language\\ 1.2.7 Extended audio description\\ 1.2.8 Media alternative\\ 1.2.9 Audio-only (Live)\end{tabular} \\ \cline{2-5} 
 &
  1.3 Adaptable &
  \begin{tabular}[c]{@{}l@{}}1.3.1 Info and relationships\\ 1.3.2 Meaningful sequence\\ 1.3.3 Sensory characteristics\end{tabular} &
  \begin{tabular}[c]{@{}l@{}}1.3.4 Orientation\\ 1.3.5 Identify input purpose\end{tabular} &
  1.3.6 Identify purpose \\ \cline{2-5} 
 &
  1.4 Distinguishable &
  \begin{tabular}[c]{@{}l@{}}1.4.1 Use of color\\ 1.4.2 Audio control\end{tabular} &
  \begin{tabular}[c]{@{}l@{}}1.4.3 Contrast (Minimum)\\ 1.4.4 Resize text\\ 1.4.5 Images of text\\ 1.4.10 Reflow\\ 1.4.11 Non-text contrast\\ 1.4.12 Text spacing\\ 1.4.13 Content on hover or \\ focus\end{tabular} &
  \begin{tabular}[c]{@{}l@{}}1.4.6 Contrast (Enhanced)\\ 1.4.7 Low or no background audio\\ 1.4.8 Visual presentation\\ 1.4.9 Images of text \\ (No exception)\end{tabular} \\ \hline
\multirow{5}{*}{\textbf{Operable}} &
  \begin{tabular}[c]{@{}l@{}}2.1 Keyboard \\       accessible\end{tabular} &
  \begin{tabular}[c]{@{}l@{}}2.1.1 Keyboard\\ 2.1.2 No keyboard trap\\ 2.1.4 Character key shortcuts\end{tabular} &
   &
  2.1.3 Keyboard (No exception) \\ \cline{2-5} 
 &
  2.2 Enough time &
  \begin{tabular}[c]{@{}l@{}}2.2.1 Timing adjustable\\ 2.2.2 Pause, stop, hide\end{tabular} &
   &
  \begin{tabular}[c]{@{}l@{}}2.2.3 No timing\\ 2.2.4 Interruptions\\ 2.2.5 Re-authenticating\\ 2.2.6 Timeouts\end{tabular} \\ \cline{2-5} 
 &
  \begin{tabular}[c]{@{}l@{}}2.3 Seizures and \\       physical reactions\end{tabular} &
  \begin{tabular}[c]{@{}l@{}}2.3.1 Three flashes or \\          below threshold\end{tabular} &
   &
  \begin{tabular}[c]{@{}l@{}}2.3.2 Three flashes\\ 2.3.3 Animation from interactions\end{tabular} \\ \cline{2-5} 
 &
  2.4 Navigable &
  \begin{tabular}[c]{@{}l@{}}2.4.1 Bypass block\\ 2.4.2 Page titled\\ 2.4.3 Focus order\\ 2.4.4 Link purpose (In context)\end{tabular} &
  \begin{tabular}[c]{@{}l@{}}2.4.5 Multiple ways\\ 2.4.6 Headings and labels\\ 2.4.7 Focus visible\\ 2.4.11 Focus not obscured \\            {[}New{]}\end{tabular} &
  \begin{tabular}[c]{@{}l@{}}2.4.8 Location\\ 2.4.9 Link purpose (Link only)\\ 2.4.10 Section headings\\ 2.4.12 Focus not obscured \\ (Enhanced) {[}New{]}\\ 2.4.13 Focus appearance {[}New{]}\end{tabular} \\ \cline{2-5} 
 &
  2.5 Input modalities &
  \begin{tabular}[c]{@{}l@{}}2.5.1 Pointer gestures\\ 2.5.2 Pointer cancellation\\ 2.5.3 Label in name\\ 2.5.4 Motion actuation\end{tabular} &
  \begin{tabular}[c]{@{}l@{}}2.5.7 Dragging movements \\ {[}New{]}\\ 2.5.8 Target size \\ (Minimum) {[}New{]}\end{tabular} &
  \begin{tabular}[c]{@{}l@{}}2.5.5 Target size (Enhanced)\\ 2.5.6 Concurrent input mechanisms\end{tabular} \\ \hline
\multirow{3}{*}{\textbf{Understandable}} &
  3.1 Readable &
  3.1.1 Language of page &
  3.1.2 Language of parts &
  \begin{tabular}[c]{@{}l@{}}3.1.3 Unusual words\\ 3.1.4 Abbrevations\\ 3.1.5 Reading level\\ 3.1.6 Pronunciation\end{tabular} \\ \cline{2-5} 
 &
  3.2 Predictable &
  \begin{tabular}[c]{@{}l@{}}3.2.1 On focus\\ 3.2.2 On input\\ 3.2.6 Consistent help {[}New{]}\end{tabular} &
  \begin{tabular}[c]{@{}l@{}}3.2.3 Consistent navigation\\ 3.2.4 Consistent identification\end{tabular} &
  3.2.5 Change on request \\ \cline{2-5} 
 &
  3.3 Input Assistance &
  \begin{tabular}[c]{@{}l@{}}3.3.1 Error identification\\ 3.3.2 Labels or instructions\\ 3.3.7 Redundant entry {[}New{]}\end{tabular} &
  \begin{tabular}[c]{@{}l@{}}3.3.3 Error suggestions\\ 3.3.4 Error prevention \\ (Legal, Financial, Data)\\ 3.3.8 Accessible authentication \\ (Minimum) {[}New{]}\end{tabular} &
  \begin{tabular}[c]{@{}l@{}}3.3.5 Help\\ 3.3.6 Error prevention (All)\\ 3.3.9 Accessible authentication \\ (Enhanced) {[}New{]}\end{tabular} \\ \hline
\textbf{Robust} &
  4.1 Compatible &
  \begin{tabular}[c]{@{}l@{}}4.1.1 Parsing \\ (Obsolete and removed)\\ 4.1.2 Name, role, value\end{tabular} &
  4.1.3 Status messages &
   \\ \hline
\end{tabular}
\end{adjustbox}
\label{table:wcag22}
\end{table}

Perceivable means that information and elements can be perceived by all users \citep{wcag22}. Non-textual content should have a textual representation, such as alternative text for images or textual description for prerecorded audio and video (guidelines 1.1 and 1.2). Adaptable describes that content is structured so that computer programs can convey the information to users regardless of device screen size or orientation. Distinguishable main purpose is for users to be able to distinguish between foreground and background information.

Operable principle ensures that the page can be used with different input peripherals, such as keyboard, mouse or touch screen \cite{wcag22}. Keyboard accessible guideline 2.1 defines that same functionality should be available by only using the keyboard. Guideline 2.2 (Enough time) requires time based content, such as notifications, to be available and controllable by the user. Seizure and physical reactions (2.3) asks designers not to create flashing content and limits the flashing of a page to three. Navigable (2.4) provides users information where they are on the page and quick access to relevant content. A known accessibility feature for navigating is the "Skip to content" link that is provided on pages for screen readers to quickly jump to the content block. The last guideline 2.5 is input modalities that guides developers to ensure functionality beyond keyboard input.

Understandable principle is about the accessibility of the content. Readable (3.1) requires web developers to set language for the page. Predictable expects that there are no sudden changes on the page, such as requesting for help button would change position. Input assistance ensures that input elements have labels, the same information would not be asked twice from the user in the same session and that input requirements and errors are displayed correctly. 

Robust requires the page to be created in such a way that it can be interpreted by assistive technologies now and in the future \citep{wcag22}. Compatible is the sole guideline defining that developers should apply roles and names for elements and that status messages can be read by assistive technologies without focusing on the element. Parsing was removed in version 2.2 as it has been addressed by other success criteria in the documents.

The success criteria under the guidelines are written as testable statements \citep{wcag22}. WCAG also offers guidance and best practices for each success criteria by providing informative techniques for web content developers. The techniques are categorized as sufficient and advisory, where advisory techniques go beyond the requirements for the specific criterion. Common failures for conformance are also documented. The techniques describing sufficient conformance are more technical and contains examples on how to implement a success criterion with HTML and CSS techniques. 

There are in total 55 success criteria to cover for conformance with AA level of web accessibility in WCAG 2.2. A comprehensive checklist \citep{wcag_checklist} of tips and techniques to achieve conformance with AA level of accessibility has been developed by the W3C.

% \section{Accessibility in web programming}

% - Talk about what the guidelines means for a web developer perspective; use WCAG 2.2 "How to meet WCAG compliance" as reference

% - 

% - Dive into some of the most basic accessibility, describe a common case, success criterion 1.1.1 for example

