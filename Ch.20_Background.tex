\chapter{Background\label{background}}

Web accessibility aims to ensure an equitable online experience for users of diverse abilities. According to World Health Organization, an estimate of over 1.3 billion people worldwide face some form of disability \citep{whodisability}. The significance of web accessibility is increasingly apparent. This chapter delve into web accessibility. In section 2.1 we describe what is web accessibility in detail, what is the current situation in form of legislation and guidelines. Section 2.2 defines the affect of web accessibility on a web developer's work. Section 2.3 is about the test automation tools.

\section{Accessibility}

Web accessibility is defined that each person has an equal opportunity to interact with web services on any device regardless their disabilities \citep{webaccessibilitydefinition}. Web designers, developers and content creators implement web accessibility. Accessibility is built upon these three topics, technical implementation, ease of use and comprehensibility of content. Technical implementation provides functionality for assistive technologies in the web source code based on web standards. Ease of use involves having a user experience with clearly structured frame for navigation and content placement. Comprehensibility of content is the structure of the content itself, that is the content is written in short paragraph and plain language. 

In web development the technical implementation is mainly done by web developers, ease of use is web designers responsibility and page content itself is written and structured by content creators. However, there are overlap in responsibilities of implementing accessibility, as creation of web pages has become easier due to website builders and different web frameworks. In the EU, the implementation details for web accessibility have been standardized based on existing guidelines.

\subsection{Legislation}

The emergence of directives and legislative measures plays a role in the goal to develop best practices for web accessibility \citep{eudirective2016}. The European Union (EU) Directive 2016/2102 was adopted in 2016 to ratify part of the United Nations Convention on the Rights of Persons with Disabilities (UN CRPD). The Directive enforces EU Member States to ensure that their public sector organizations implement and monitor accessibility on their web pages and mobile applications. The directive gave a clear timeline for the member states. At latest of September 2018 each member state should have transposed this directive into national legislation. Since September 2019 all new web sites must conform. All public sector websites had to comply after September 2020. In June 2021 all mobile applications had to conform and in December 2021 the monitoring and reporting started. The European Accessibility Act (EAA) was adopted by the EU in 2019 to complement the 2016 directive and UN CRPD \citep{eudirective2019}. The EAA will bring similar requirements for private sector bodies starting from end of June 2025.

The EU Directive adopted in 2016 refers to the European Standard \textit{EN 301 549} that specifies the accessibility requirements for ICT products and services. In the first version of the Standard the guidelines and requirements were based on Web Content Accessibility Guidelines (WCAG) 2.0 provided by the World Wide Web Consortium \citep{wcagadoptioneurope}. During the same time as member states had to transpose this into a national law in 2018, the standard was changed to be a direct reference to the WCAG version 2.1 making the WCAG 2.1 the accessibility standard that member states should follow. 


\subsection{Guidelines}

The Web Content Accessibility Guidelines (WCAG) is created by accessibility specialists at the World Wide Web Consortium \citep{wcagstandard}. Version 2 was published in 2008, and later updated in 2018 with

- Latest guidelines, WCAG 2.2. Write about the 4 principles; perceivable, operable, understandable and robust. Create a table to describe these 4 principles, the success criteria, the classification required by law (A, AA, AAA).


\section{Accessibility in web programming}

- Talk about what the guidelines means for a web developer perspective; use WCAG 2.2 "How to meet WCAG compliance" as reference

- Dive into some of the most basic accessibility, what it means
\section{Accessibility evaluation}

- Write about how accessibility can be tested. Automated/manual requirements.

\subsection{Tools}

- https://dl-acm-org.libproxy.helsinki.fi/doi/pdf/10.1145/3563137.3563148

- Monitoring with automated testing tools; https://eur-lex.europa.eu/legal-content/EN/TXT/HTML/?uri=CELEX:32018D1524

\section{Manual / Beyond reach}