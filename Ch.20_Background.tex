\chapter{Background\label{background}}

Web accessibility aims to ensure an equitable online experience for users of diverse abilities. According to World Health Organization, an estimate of over 1.3 billion people worldwide face some form of disability \citep{whodisability}. The significance of web accessibility is increasingly apparent. This chapter delve into web accessibility. In section 2.1 we describe what is web accessibility in detail, what is the current situation in form of legislation and guidelines. Section 2.2 defines the affect of web accessibility on a web developer's work. Section 2.3 is about the test automation tools.

\section{Accessibility}

Web accessibility is defined that each person has an equal opportunity to interact with web services on any device regardless their disabilities \citep{webaccessibilitydefinition}. Web designers, developers and content creators implement web accessibility. Accessibility is built upon these three topics, technical implementation, ease of use and comprehensibility of content. Technical implementation provides functionality for assistive technologies in the web source code based on web standards. Ease of use involves having a user experience with clearly structured frame for navigation and content placement. Comprehensibility of content is the structure of the content itself, that is the content is written in short paragraph and plain language. 

In web development the technical implementation is mainly done by web developers, ease of use is web designers responsibility and page content itself is written and structured by content creators. However, there are overlap in responsibilities of implementing accessibility, as creation of web pages has become easier due to website builders and different web frameworks. In the EU, the implementation details for web accessibility have been standardized based on existing guidelines.

\subsection{Legislation}

The emergence of directives and legislative measures plays a role in the goal to develop best practices for web accessibility \citep{eudirective2016}. The European Union (EU) Directive 2016/2102 was adopted in 2016 to ratify part of the United Nations Convention on the Rights of Persons with Disabilities (UN CRPD). The Directive enforces EU Member States to ensure that their public sector organizations implement and monitor accessibility on their web pages and mobile applications. The directive gave a clear timeline for the member states. At latest of September 2018 each member state should have transposed this directive into national legislation. Since September 2019 all new web sites must conform. All public sector websites had to comply after September 2020. In June 2021 all mobile applications had to conform and in December 2021 the monitoring and reporting started. The European Accessibility Act (EAA) was adopted by the EU in 2019 to complement the 2016 directive and UN CRPD \citep{eudirective2019}. The EAA will bring similar requirements for private sector bodies starting from end of June 2025. However, currently there has been no decision made if \cite{eudirective2019} will use the same standard as \cite{eudirective2016}.

The EU Directive adopted in 2016 refers to the European Standard \textit{EN 301 549} that specifies the accessibility requirements for ICT products and services. In the first version of the Standard the guidelines and requirements were based on Web Content Accessibility Guidelines (WCAG) 2.0 provided by the World Wide Web Consortium \citep{wcagadoptioneurope}. During the same time as member states had to transpose this into a national law in 2018, the standard was changed to be a direct reference to the WCAG version 2.1 making the WCAG 2.1 the accessibility standard that member states should follow. There are three levels of accessibility, A being the lowest, AA and AAA being hte highest. The directive requires the public sector to conform with AA level of the guidelines.


\subsection{Guidelines}

Web Content Accessibility Guidelines (WCAG) is created by accessibility specialists in a working group at the World Wide Web Consortium \citep{wcagoverview}. Version 2 was published in 2008. Version 2.1 came out in 2018 and is the version referenced in the European Standard. Version 2.2 was published and made a recommendation by W3C in October 2023. The consortium recommends always to use the latest version as they are backwards compatible. Therefore, in this thesis we will focus on version 2.2.

WCAG guidelines aim to ensure that content is accessible to a broad audience and adaptable to various forms, accommodating diverse sensory, physical, and cognitive abilities of individuals \citep{wcag22}. It is based on four principles: \textit{perceivable, operable, understandable, and robust}. In each principle, there are guidelines and under each guideline are success criteria. There are a total of 13 guidelines and 86 success criteria. In total there are three levels of accessibility; A being the lowest, AA, and AAA being the highest, see table \ref{table:wcag22}. Each success criterion is part of one level of accessibility. Version 2.2 would remove one and add six new success criteria for public sector to implement when the EU standard is updated to reflect the changes. Success criterion 4.1.1 is made obsolete and the six added success criteria are: 2.4.11, 2.5.7, 2.5.8, 3.2.6, 3.3.7 and 3.3.8 as shown in table \ref{table:wcag22}.

% Please add the following required packages to your document preamble:
% \usepackage{multirow}
% \usepackage[normalem]{ulem}
% \useunder{\uline}{\ul}{}
\begin{table}[]
\caption{WCAG 2.2 principles, guidelines and success criteria divided with the three accessibility levels. Adapted from \textcite{wcag22}}
\begin{adjustbox}{width=1\textwidth}
\begin{tabular}{|l|l|l|l|l|}
\hline
\textbf{Principle} &
  \multicolumn{1}{c|}{\textbf{Guideline}} &
  \multicolumn{1}{c|}{\textbf{\begin{tabular}[c]{@{}c@{}}Success Criterion\\ Level A\end{tabular}}} &
  \multicolumn{1}{c|}{\textbf{\begin{tabular}[c]{@{}c@{}}Success Criterion\\ Level AA\end{tabular}}} &
  \multicolumn{1}{c|}{\textbf{\begin{tabular}[c]{@{}c@{}}Success Criterion\\ Level AAA\end{tabular}}} \\ \hline
\multirow{4}{*}{\textbf{Perceivable}} &
  1.1 Text alternatives &
  1.1.1 Non-text content &
   &
   \\ \cline{2-5} 
 &
  \begin{tabular}[c]{@{}l@{}}1.2 Time-based media \\ (Prerecorded, \\ unless stated otherwise)  \end{tabular} &
  \begin{tabular}[c]{@{}l@{}}1.2.1 Audio-only and \\ Video-only\\ 1.2.2 Captions\\ 1.2.3 Audio description \\ or media alternative\end{tabular} &
  \begin{tabular}[c]{@{}l@{}}1.2.4 Captions (Live)\\ 1.2.5 Audio description\end{tabular} &
  \begin{tabular}[c]{@{}l@{}}1.2.6 Sign language\\ 1.2.7 Extended audio description\\ 1.2.8 Media alternative\\ 1.2.9 Audio-only (Live)\end{tabular} \\ \cline{2-5} 
 &
  1.3 Adaptable &
  \begin{tabular}[c]{@{}l@{}}1.3.1 Info and relationships\\ 1.3.2 Meaningful sequence\\ 1.3.3 Sensory characteristics\end{tabular} &
  \begin{tabular}[c]{@{}l@{}}1.3.4 Orientation\\ 1.3.5 Identify input purpose\end{tabular} &
  1.3.6 Identify purpose \\ \cline{2-5} 
 &
  1.4 Distinguishable &
  \begin{tabular}[c]{@{}l@{}}1.4.1 Use of color\\ 1.4.2 Audio control\end{tabular} &
  \begin{tabular}[c]{@{}l@{}}1.4.3 Contrast (Minimum)\\ 1.4.4 Resize text\\ 1.4.5 Images of text\\ 1.4.10 Reflow\\ 1.4.11 Non-text contrast\\ 1.4.12 Text spacing\\ 1.4.13 Content on hover or \\ focus\end{tabular} &
  \begin{tabular}[c]{@{}l@{}}1.4.6 Contrast (Enhanced)\\ 1.4.7 Low or no background audio\\ 1.4.8 Visual presentation\\ 1.4.9 Images of text \\ (No exception)\end{tabular} \\ \hline
\multirow{5}{*}{\textbf{Operable}} &
  \begin{tabular}[c]{@{}l@{}}2.1 Keyboard \\       accessible\end{tabular} &
  \begin{tabular}[c]{@{}l@{}}2.1.1 Keyboard\\ 2.1.2 No keyboard trap\\ 2.1.4 Character key shortcuts\end{tabular} &
   &
  2.1.3 Keyboard (No exception) \\ \cline{2-5} 
 &
  2.2 Enough time &
  \begin{tabular}[c]{@{}l@{}}2.2.1 Timing adjustable\\ 2.2.2 Pause, stop, hide\end{tabular} &
   &
  \begin{tabular}[c]{@{}l@{}}2.2.3 No timing\\ 2.2.4 Interruptions\\ 2.2.5 Re-authenticating\\ 2.2.6 Timeouts\end{tabular} \\ \cline{2-5} 
 &
  \begin{tabular}[c]{@{}l@{}}2.3 Seizures and \\       physical reactions\end{tabular} &
  \begin{tabular}[c]{@{}l@{}}2.3.1 Three flashes or \\          below threshold\end{tabular} &
   &
  \begin{tabular}[c]{@{}l@{}}2.3.2 Three flashes\\ 2.3.3 Animation from interactions\end{tabular} \\ \cline{2-5} 
 &
  2.4 Navigable &
  \begin{tabular}[c]{@{}l@{}}2.4.1 Bypass block\\ 2.4.2 Page titled\\ 2.4.3 Focus order\\ 2.4.4 Link purpose (In context)\end{tabular} &
  \begin{tabular}[c]{@{}l@{}}2.4.5 Multiple ways\\ 2.4.6 Headings and labels\\ 2.4.7 Focus visible\\ 2.4.11 Focus not obscured \\            {[}New{]}\end{tabular} &
  \begin{tabular}[c]{@{}l@{}}2.4.8 Location\\ 2.4.9 Link purpose (Link only)\\ 2.4.10 Section headings\\ 2.4.12 Focus not obscured \\ (Enhanced) {[}New{]}\\ 2.4.13 Focus appearance {[}New{]}\end{tabular} \\ \cline{2-5} 
 &
  2.5 Input modalities &
  \begin{tabular}[c]{@{}l@{}}2.5.1 Pointer gestures\\ 2.5.2 Pointer cancellation\\ 2.5.3 Label in name\\ 2.5.4 Motion actuation\end{tabular} &
  \begin{tabular}[c]{@{}l@{}}2.5.7 Dragging movements \\ {[}New{]}\\ 2.5.8 Target size \\ (Minimum) {[}New{]}\end{tabular} &
  \begin{tabular}[c]{@{}l@{}}2.5.5 Target size (Enhanced)\\ 2.5.6 Concurrent input mechanisms\end{tabular} \\ \hline
\multirow{3}{*}{\textbf{Understandable}} &
  3.1 Readable &
  3.1.1 Language of page &
  3.1.2 Language of parts &
  \begin{tabular}[c]{@{}l@{}}3.1.3 Unusual words\\ 3.1.4 Abbrevations\\ 3.1.5 Reading level\\ 3.1.6 Pronunciation\end{tabular} \\ \cline{2-5} 
 &
  3.2 Predictable &
  \begin{tabular}[c]{@{}l@{}}3.2.1 On focus\\ 3.2.2 On input\\ 3.2.6 Consistent help {[}New{]}\end{tabular} &
  \begin{tabular}[c]{@{}l@{}}3.2.3 Consistent navigation\\ 3.2.4 Consistent identification\end{tabular} &
  3.2.5 Change on request \\ \cline{2-5} 
 &
  3.3 Input Assistance &
  \begin{tabular}[c]{@{}l@{}}3.3.1 Error identification\\ 3.3.2 Labels or instructions\\ 3.3.7 Redundant entry {[}New{]}\end{tabular} &
  \begin{tabular}[c]{@{}l@{}}3.3.3 Error suggestions\\ 3.3.4 Error prevention \\ (Legal, Financial, Data)\\ 3.3.8 Accessible authentication \\ (Minimum) {[}New{]}\end{tabular} &
  \begin{tabular}[c]{@{}l@{}}3.3.5 Help\\ 3.3.6 Error prevention (All)\\ 3.3.9 Accessible authentication \\ (Enhanced) {[}New{]}\end{tabular} \\ \hline
\textbf{Robust} &
  4.1 Compatible &
  \begin{tabular}[c]{@{}l@{}}4.1.1 Parsing \\ (Obsolete and removed)\\ 4.1.2 Name, role, value\end{tabular} &
  4.1.3 Status messages &
   \\ \hline
\end{tabular}
\end{adjustbox}
\label{table:wcag22}
\end{table}

- Perceivable

- Operable

- Understandable

- Robust


\section{Accessibility in web programming}

- Talk about what the guidelines means for a web developer perspective; use WCAG 2.2 "How to meet WCAG compliance" as reference

- Dive into some of the most basic accessibility, what it means

\section{Accessibility evaluation}

- Write about how accessibility can be tested. Automated/manual requirements.

\subsection{Tools}

- https://dl-acm-org.libproxy.helsinki.fi/doi/pdf/10.1145/3563137.3563148

- Monitoring with automated testing tools; https://eur-lex.europa.eu/legal-content/EN/TXT/HTML/?uri=CELEX:32018D1524

\section{Manual / Beyond reach}