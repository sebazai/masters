\documentclass[english,twoside,censored,csm,software-track-2020]{HYthesisML} 

\PassOptionsToClass{openany,twoside,a4paper}{report}

\usepackage{csquotes}
\usepackage[style=authoryear,bibstyle=authoryear,backend=biber,natbib=true,maxnames=99,maxcitenames=2,uniquelist=minyear,giveninits=true,uniquename=mininit,urldate=long]{biblatex}

\addbibresource{bibliography.bib}
\DeclareNameAlias{sortname}{family-given}


\usepackage{lmodern}        
\usepackage{textcomp}      
\usepackage[pdftex]{color, graphicx}
\usepackage{epsfig}
\usepackage{subfigure}

\usepackage[pdftex, plainpages=false]{hyperref} 
\usepackage{fancyhdr}
\usepackage{tikz}
\usepackage{amsmath, amssymb}

\usepackage[normalem]{ulem}
\useunder{\uline}{\ul}{}
\usepackage{multirow}
\usepackage{adjustbox}
\usepackage{listings}
\lstset{language=HTML, basicstyle=\footnotesize, breaklines=true, tabsize=1}

\usepackage{pifont}
\newcommand{\cmark}{\ding{51}}  

\singlespacing 

\fussy
\overfullrule=1mm
\showboxbreadth=50 
\showboxdepth=50



\title{Capabilities of Large Language Models in Web Accessibility Evaluation: A Design Science Approach}

\author{Sebastian Sergelius}
\date{\today}

% Set supervisors, use the titles according to the thesis language
% in English Prof. or Dr., or in Finnish toht. or tri or FT, TkT, Ph.D. or in Swedish... 
\supervisors{Prof. Tomi Männistö}

\keywords{web accessibility, accessibility evaluation}
\additionalinformation{\translate{\track}}

%% For seminar reports:
%%\additionalinformation{Name of the seminar}

%% Provide classification terms, to appear on the abstract page.
%% Replace the classification terms below with the ones that match your work.
%% ACM Digital library provides a taxonomy and a tool for classification
%% in computer science. Use 1-3 paths, and use right arrows between the
%% about three levels in the path; each path requires a new line.

\classification{\protect{\ \\
\  Human-centered computing $\rightarrow$ Accessibility  $\rightarrow$ Accessibility systems and tools\  \\
\  Human-centered computing $\rightarrow$ Accessibility  $\rightarrow$ Accessibility technologies\  \\
}}

%% If you want to quote someone special. You can comment this line out and there will be nothing on the document.
%\quoting{Bachelor's degrees make pretty good placemats if you get them laminated.}{Jeph Jacques}


%% OPTIONAL STEP: Set up properties and metadata for the pdf file that pdfLaTeX makes.
%% Your name, work title, and keywords are recommended.
\hypersetup{
    unicode=true,           % to show non-Latin characters in Acrobat’s bookmarks
    pdftoolbar=true,        % show Acrobat’s toolbar?
    pdfmenubar=true,        % show Acrobat’s menu?
    pdffitwindow=false,     % window fit to page when opened
    pdfstartview={FitH},    % fits the width of the page to the window
    pdftitle={},            % title
    pdfauthor={},           % author
    pdfsubject={},          % subject of the document
    pdfcreator={},          % creator of the document
    pdfproducer={pdfLaTeX}, % producer of the document
    pdfkeywords={something} {something else}, % list of keywords for
    pdfnewwindow=true,      % links in new window
    colorlinks=true,        % false: boxed links; true: colored links
    linkcolor=black,        % color of internal links
    citecolor=black,        % color of links to bibliography
    filecolor=magenta,      % color of file links
    urlcolor=cyan           % color of external links
}

%%-----------------------------------------------------------------------------------

\begin{document}

% Generate title page.
\maketitle

%%%%%%%%%%%%%%%%%%%%%%%%%%%%%%%%%%%%%%%%%%%%%%%%%%%%%%%%%
%% STEP 3:
%%%%%%%%%%%%%%%%%%%%%%%%%%%%%%%%%%%%%%%%%%%%%%%%%%%%%%%%%
%% Write your abstract in the separate file, to be positioned here.
%% You can make several abstract pages (if you want it in different languages),
%% in which case you should also define the language of the abstract,
%% as below.

\begin{otherlanguage}{english}
\begin{abstract}

Web accessibility is an issue on many websites, hindering people with disabilities to perceive information on the web. Accessibility evaluation based on existing accessibility guidelines helps to find the barriers that affect these users. A comprehensive accessibility conformance review requires manual labor. However, Accessibility Evaluation Tools (AET) help find the most common accessibility barriers. This thesis explores the potential of Large Language Models (LLM) in evaluating the Web Content Accessibility Guideline (WCAG) success criterion 2.4.2, Page Titled, using pre-made HTML test cases provided by the WCAG Accessibility Conformance Testing rules. Utilizing a Design Science Research method, the input prompt is iterated to evaluate whether the LLM can accurately assess if the HTML code contains a title, whether the title describes the page content, and whether the title identifies the page. The findings reveal that the LLM can perform these evaluations, suggesting that it can be used as an assistant in conformance reviews, potentially speeding up the conformance review process and reducing the need to understand the website's subject matter. However, the study acknowledges the simplicity of the test cases and the non-deterministic nature of LLMs. Future research on LLMs evaluating web accessibility should address language diversity. In addition, implementing LLMs into AETs would provide insight into how LLMs affect the efficiency of the accessibility evaluation process and the capabilities of LLMs on more complex websites.
\end{abstract}
\end{otherlanguage}

% Place ToC
%\newpage
\mytableofcontents

\mainmatter

%%%%%%%%%%%%%%%%%%%%%%%%%%%%%%%%%%%%%%%%%%%%%%%%%%%%%%%%%
%% STEP 4: Write the thesis.
%%%%%%%%%%%%%%%%%%%%%%%%%%%%%%%%%%%%%%%%%%%%%%%%%%%%%%%%%
%% Your actual text starts here. You shouldn't mess with the code above the line except
%% to change the parameters. Removing the abstract and ToC commands will mess up stuff.
%%
%% Command \include{file} includes the file of name file.tex.
%% A new page will be created at every \include command, 
%% which makes it appropriate to use it for large entities such as book chapters. Cannot be nested.
%% It is useful for a big project, as changing one of the include targets 
%% won't force the regeneration of the outputs of all the rest.
%% Alternatively, \input is a more lower level macro 
%% which simply inputs the content of the given file like it was copy&pasted there manually.

\chapter{Introduction\label{intro}}

For the past decade, web accessibility legislation in the EU has enforced the public sector websites to comply with the web accessibility guidelines \citep{eudirective2016}. Additionally, the European Accessibility Act will enforce some of the private sector websites to comply as well at the end of June 2025 \citep{eudirective2019}. The purpose of the legislation is to provide equal access to digital information and services for people with disabilities.  
\chapter{Web accessibility\label{accessibility}}

Web accessibility aims to ensure an equitable online experience for users of diverse abilities. According to World Health Organization, an estimate of over 1.3 billion people worldwide face some form of significant disability \citep{whodisability}. Persons with low vision, hard of hearing or learning disability might have difficulties perceiving information on the web. The significance of web accessibility is increasingly apparent. This chapter delve into web accessibility foundations. 

Section 2.1 defines web accessibility. In section 2.2 we describe commonly known disabilities affecting the use of web services. Section 2.3 on how to implement web accessibility and what are the web standards used to overcome barriers. Section 2.4 delves into the international standard for web accessibility called Web Content Accessibility Guidelines (WCAG). Last section of this chapter are the legal requirements for web accessibility today in the European Union.

\section{Definition}

Web accessibility is defined as an equal opportunity to perceive, operate, understand and interact with an online service \citep{webaccessibilitydefinition}. Regardless of disability a person faces, the same information and functionality should be made available. Web accessibility considers the technical and content implementation. Technical implementation of web accessibility ensures that barriers for assistive technologies, such as screen readers or speech input software, are removed. Structuring content with headings, using short paragraphs and simple understandable language makes the content accessible. Additionally, content accessibility is providing content in multiple forms, such as videos, sound, images and text.

\section{Disabilities affecting the use of web}

Spanning from visual impairments to situational limitations, each condition presents barriers that hinder the ability to interact and comprehend information on the web. Understanding these disabilities is crucial when designing online services that prioritize inclusion and ensures an equal experience for all users. Disabilities that affect using the web are vision, physical, auditory and cognitive \citep{w3cbarriers}.

One of the most commonly known condition affecting web accessibility is vision impairment and blindness with more then 2.2 billion people impacted worldwide \citep[Chapter~1]{whovision, webaccessibility}. Individuals with visual impairments depend on functionality for formatting the content to their needs and on the content accessibility. Barriers encountered by individuals with visual impairment are lack of text alternatives for images ans controls, insufficient contrast between text color and background color, lack of keyboard navigation support and missing descriptions for links and inputs. 

Accessibility is not only targeted for individuals with impairments. Individuals that have lost their glasses, has a broken arm or a hard time reading the screen when the sun glares also benefits of web accessibility \citep{w3cbarriers}. In addition to temporal disabilities and situational limitations, also aging people benefit. Most of the people with low vision are above the age of 50 \citep{whovision}.

Since 2019, \textcite{webaimmillions} has conducted an annual research analysing accessibility of the top million web sites. In their research they used their own WAVE automated accessibility evaluation tool to check for accessibility barriers on the landing page. Results show that 95.9\% of landing pages have accessibility barriers. In addition, the amount of elements and ARIA tags of web pages has grown each year indicating that web pages are more complex then before. Moreover, the five most found accessibility barriers in the past 5 years all affect people with low vision or blindness using assistive technologies and keyboard only \citep{webaimmillions}. 

\section{Implementation and standards}

Web accessibility strives for an equal opportunity for persons with any sort of impairment to interact with web services on any device with help of assistive technologies. In web accessibility the goal is to implement a web page in a specific way so that assistive technologies can help to remove the barrier for accessing the information provided. Commonly known assistive technologies are screen readers, screen magnifiers and speech input software that are incorporated in major operating systems.

Web designers, developers and content creators implement web accessibility. Accessibility is built upon these three topics: technical implementation, ease of use and comprehensibility of content \citep{webaccessibilitydefinition}. Technical implementation provides functionality for assistive technologies in the web source code based on web standards. Ease of use involves having a user experience with clearly structured frame for navigation and content placement. Comprehensibility of content is the structure of the content itself, meaning the content is written in short paragraphs and plain language. 

In web development the technical implementation is mainly done by web developers by following the HTML and CSS standard. Accessible Rich Internet Application (ARIA) is a standard used within HTML for advanced web applications with dynamic content to help screen readers interpret the content. For example, removing the barrier for individuals with low vision to perceive images requires an alternative text in the image tag in HTML. In addition, having a progress bar on a web application requires ARIA tags for screen readers to communicate the status of an action.

However, there are overlap in responsibilities of implementing accessibility, as building and managing web pages has become easier due to website builders and content management systems. In the EU, the implementation details for web accessibility have been standardized based on existing guidelines.

\section{Guidelines}

The World Wide Web Consortium creates accessibility guidelines through the Web Accessibility Initiative \citep{wcagoverview}. The international standard is named Web Accessibility Content Guidelines (WCAG). Version 2 of WCAG was published in 2008. Version 2.1 came out in 2018 and is the version referenced in the European Standard. Version 2.2 was published, and made a recommendation by W3C in October 2023. The W3C recommends always to use the latest version as they are backwards compatible. 

The guidelines aim to ensure that digital content is accessible to a broad audience and adaptable to various forms, accommodating diverse sensory, physical, and cognitive abilities of individuals \citep{wcag22}. The WCAG document is written in a technology agnostic way. It is based on four principles: \textit{perceivable, operable, understandable, and robust}. In each principle, there are guidelines and under each guideline are success criteria. There are a total of 13 guidelines and 86 success criteria. In total there are three levels of accessibility; A being the lowest, AA, and AAA being the highest, see table \ref{table:wcag22}. 

Each success criterion is part of one level of accessibility. Version 2.2 remove one and add nine new success criteria. Success criterion 4.1.1 is made obsolete, as assistive technologies do not need to directly parse the HTML code anymore, and problems occurred by this criterion is addressed by other criteria. The new nine added success criteria are: 2.4.11, 2.4.12, 2.4.13, 2.5.7, 2.5.8, 3.2.6, 3.3.7, 3.3.8 and 3.3.9 as shown in Table \ref{table:wcag22}. 

% Please add the following required packages to your document preamble:
% \usepackage{multirow}
% \usepackage[normalem]{ulem}
% \useunder{\uline}{\ul}{}
\begin{table}[]
\caption{WCAG 2.2 principles, guidelines and success criteria divided with the three accessibility levels. Adapted from \textcite{wcag22}}
\begin{adjustbox}{width=1\textwidth}
\begin{tabular}{|l|l|l|l|l|}
\hline
\textbf{Principle} &
  \multicolumn{1}{c|}{\textbf{Guideline}} &
  \multicolumn{1}{c|}{\textbf{\begin{tabular}[c]{@{}c@{}}Success Criterion\\ Level A\end{tabular}}} &
  \multicolumn{1}{c|}{\textbf{\begin{tabular}[c]{@{}c@{}}Success Criterion\\ Level AA\end{tabular}}} &
  \multicolumn{1}{c|}{\textbf{\begin{tabular}[c]{@{}c@{}}Success Criterion\\ Level AAA\end{tabular}}} \\ \hline
\multirow{4}{*}{\textbf{Perceivable}} &
  1.1 Text alternatives &
  1.1.1 Non-text content &
   &
   \\ \cline{2-5} 
 &
  \begin{tabular}[c]{@{}l@{}}1.2 Time-based media \\ (Prerecorded, \\ unless stated otherwise)  \end{tabular} &
  \begin{tabular}[c]{@{}l@{}}1.2.1 Audio-only and \\ Video-only\\ 1.2.2 Captions\\ 1.2.3 Audio description \\ or media alternative\end{tabular} &
  \begin{tabular}[c]{@{}l@{}}1.2.4 Captions (Live)\\ 1.2.5 Audio description\end{tabular} &
  \begin{tabular}[c]{@{}l@{}}1.2.6 Sign language\\ 1.2.7 Extended audio description\\ 1.2.8 Media alternative\\ 1.2.9 Audio-only (Live)\end{tabular} \\ \cline{2-5} 
 &
  1.3 Adaptable &
  \begin{tabular}[c]{@{}l@{}}1.3.1 Info and relationships\\ 1.3.2 Meaningful sequence\\ 1.3.3 Sensory characteristics\end{tabular} &
  \begin{tabular}[c]{@{}l@{}}1.3.4 Orientation\\ 1.3.5 Identify input purpose\end{tabular} &
  1.3.6 Identify purpose \\ \cline{2-5} 
 &
  1.4 Distinguishable &
  \begin{tabular}[c]{@{}l@{}}1.4.1 Use of color\\ 1.4.2 Audio control\end{tabular} &
  \begin{tabular}[c]{@{}l@{}}1.4.3 Contrast (Minimum)\\ 1.4.4 Resize text\\ 1.4.5 Images of text\\ 1.4.10 Reflow\\ 1.4.11 Non-text contrast\\ 1.4.12 Text spacing\\ 1.4.13 Content on hover or \\ focus\end{tabular} &
  \begin{tabular}[c]{@{}l@{}}1.4.6 Contrast (Enhanced)\\ 1.4.7 Low or no background audio\\ 1.4.8 Visual presentation\\ 1.4.9 Images of text \\ (No exception)\end{tabular} \\ \hline
\multirow{5}{*}{\textbf{Operable}} &
  \begin{tabular}[c]{@{}l@{}}2.1 Keyboard \\       accessible\end{tabular} &
  \begin{tabular}[c]{@{}l@{}}2.1.1 Keyboard\\ 2.1.2 No keyboard trap\\ 2.1.4 Character key shortcuts\end{tabular} &
   &
  2.1.3 Keyboard (No exception) \\ \cline{2-5} 
 &
  2.2 Enough time &
  \begin{tabular}[c]{@{}l@{}}2.2.1 Timing adjustable\\ 2.2.2 Pause, stop, hide\end{tabular} &
   &
  \begin{tabular}[c]{@{}l@{}}2.2.3 No timing\\ 2.2.4 Interruptions\\ 2.2.5 Re-authenticating\\ 2.2.6 Timeouts\end{tabular} \\ \cline{2-5} 
 &
  \begin{tabular}[c]{@{}l@{}}2.3 Seizures and \\       physical reactions\end{tabular} &
  \begin{tabular}[c]{@{}l@{}}2.3.1 Three flashes or \\          below threshold\end{tabular} &
   &
  \begin{tabular}[c]{@{}l@{}}2.3.2 Three flashes\\ 2.3.3 Animation from interactions\end{tabular} \\ \cline{2-5} 
 &
  2.4 Navigable &
  \begin{tabular}[c]{@{}l@{}}2.4.1 Bypass block\\ 2.4.2 Page titled\\ 2.4.3 Focus order\\ 2.4.4 Link purpose (In context)\end{tabular} &
  \begin{tabular}[c]{@{}l@{}}2.4.5 Multiple ways\\ 2.4.6 Headings and labels\\ 2.4.7 Focus visible\\ 2.4.11 Focus not obscured \\            {[}New{]}\end{tabular} &
  \begin{tabular}[c]{@{}l@{}}2.4.8 Location\\ 2.4.9 Link purpose (Link only)\\ 2.4.10 Section headings\\ 2.4.12 Focus not obscured \\ (Enhanced) {[}New{]}\\ 2.4.13 Focus appearance {[}New{]}\end{tabular} \\ \cline{2-5} 
 &
  2.5 Input modalities &
  \begin{tabular}[c]{@{}l@{}}2.5.1 Pointer gestures\\ 2.5.2 Pointer cancellation\\ 2.5.3 Label in name\\ 2.5.4 Motion actuation\end{tabular} &
  \begin{tabular}[c]{@{}l@{}}2.5.7 Dragging movements \\ {[}New{]}\\ 2.5.8 Target size \\ (Minimum) {[}New{]}\end{tabular} &
  \begin{tabular}[c]{@{}l@{}}2.5.5 Target size (Enhanced)\\ 2.5.6 Concurrent input mechanisms\end{tabular} \\ \hline
\multirow{3}{*}{\textbf{Understandable}} &
  3.1 Readable &
  3.1.1 Language of page &
  3.1.2 Language of parts &
  \begin{tabular}[c]{@{}l@{}}3.1.3 Unusual words\\ 3.1.4 Abbrevations\\ 3.1.5 Reading level\\ 3.1.6 Pronunciation\end{tabular} \\ \cline{2-5} 
 &
  3.2 Predictable &
  \begin{tabular}[c]{@{}l@{}}3.2.1 On focus\\ 3.2.2 On input\\ 3.2.6 Consistent help {[}New{]}\end{tabular} &
  \begin{tabular}[c]{@{}l@{}}3.2.3 Consistent navigation\\ 3.2.4 Consistent identification\end{tabular} &
  3.2.5 Change on request \\ \cline{2-5} 
 &
  3.3 Input Assistance &
  \begin{tabular}[c]{@{}l@{}}3.3.1 Error identification\\ 3.3.2 Labels or instructions\\ 3.3.7 Redundant entry {[}New{]}\end{tabular} &
  \begin{tabular}[c]{@{}l@{}}3.3.3 Error suggestions\\ 3.3.4 Error prevention \\ (Legal, Financial, Data)\\ 3.3.8 Accessible authentication \\ (Minimum) {[}New{]}\end{tabular} &
  \begin{tabular}[c]{@{}l@{}}3.3.5 Help\\ 3.3.6 Error prevention (All)\\ 3.3.9 Accessible authentication \\ (Enhanced) {[}New{]}\end{tabular} \\ \hline
\textbf{Robust} &
  4.1 Compatible &
  \begin{tabular}[c]{@{}l@{}}4.1.1 Parsing \\ (Obsolete and removed)\\ 4.1.2 Name, role, value\end{tabular} &
  4.1.3 Status messages &
   \\ \hline
\end{tabular}
\end{adjustbox}
\label{table:wcag22}
\end{table}

Perceivable means that information and elements can be perceived by all users \citep{wcag22}. Non-textual content should have a textual representation, such as alternative text for images or textual description for prerecorded audio and video (guidelines 1.1 and 1.2). Adaptable describes that content is structured so that computer programs can convey the information to users regardless of device screen size or orientation. Distinguishable main purpose is for users to be able to distinguish between foreground and background information.

Operable principle ensures that the page can be used with different input peripherals, such as keyboard, mouse or touch screen \citep{wcag22}. Keyboard accessible guideline 2.1 defines that same functionality should be available by only using the keyboard. Guideline 2.2 (Enough time) requires time based content, such as notifications, to be available and controllable by the user. Seizure and physical reactions (2.3) asks designers not to create flashing content and limits the flashing of a page to three. Navigable (2.4) provides users information where they are on the page and quick access to relevant content. A known accessibility feature for navigating is the "Skip to content" link that is provided on pages for screen readers to quickly jump to the content block. The last guideline 2.5 is input modalities that guides developers to ensure functionality beyond keyboard input.

Understandable principle is about the accessibility of the content. Readable (3.1) requires web developers to set language for the page. Predictable expects that there are no sudden changes on the page, such as requesting for help button would change position. Input assistance ensures that input elements have labels, the same information would not be asked twice from the user in the same session and that input requirements and errors are displayed correctly. 

Robust requires the page to be created in such a way that it can be interpreted by assistive technologies now and in the future \citep{wcag22}. Compatible is the sole guideline defining that developers should apply roles and names for elements and that status messages can be read by assistive technologies without focusing on the element. Parsing was removed in version 2.2 as it has been addressed by other success criteria in the documents.

The success criteria under the guidelines are written as testable statements \citep{wcag22}. WCAG also offers guidance and best practices for each success criteria by providing informative techniques for web content developers. The techniques are categorized as sufficient and advisory, where advisory techniques go beyond the requirements for the specific criterion. Common failures for conformance are also documented. The techniques describing sufficient conformance are more technical and contains examples on how to implement a success criterion with HTML and CSS techniques. 

There are in total 55 success criteria to cover for conformance with AA level of web accessibility in WCAG 2.2. A comprehensive checklist \citep{wcag_checklist} of tips and techniques to achieve conformance with AA level of accessibility has been developed by the W3C. The AA level of web accessibility is what legislation requires from the public sector and in future for some of the private sector.  Out of the nine new success criteria in WCAG 2.2, there will be six new to be implemented to achieve AA-level of accessibility.


\section{Legislation}

The emergence of directives and legislative measures plays a role in the goal to develop best practices for web accessibility. The European Union (EU) Directive 2016/2102 was adopted in 2016 to ratify part of the United Nations Convention on the Rights of Persons with Disabilities (UN CRPD). The Directive enforces EU Member States to ensure that their public sector organizations implement and monitor accessibility on their web pages and mobile applications \citep{eudirective2016}. 

\textcite{eudirective2016} gave a clear timeline for the member states. At latest of September 2018 each member state should have transposed this directive into national legislation. Since September 2019 all new web sites must conform. All public sector websites had to comply after September 2020. In June 2021 mobile applications had to conform and in December 2021 the monitoring and reporting started. 

The EU Directive adopted in 2016 refers to the European Standard EN 301 549 that specifies the accessibility requirements for ICT products and services. In the first version of the Standard the guidelines and requirements were based on Web Content Accessibility Guidelines provided by the World Wide Web Consortium \citep{wcagadoptioneurope}. During the same time as member states had to transpose this into a national law in 2018, the standard was changed to be a direct reference to the WCAG making the WCAG an accessibility standard that EU member states should follow. The EU commission is in progress of harmonising the EN 301 549 Standard with the latest WCAG version with a planned release of 2025 \citep{etsi_standard}.

Monitoring of accessibility in the public sector should be done every three years \citep{eudirectivemonitoring}. The first monitoring and accessibility report from member states to the EU commission was given in 2021. Next reports should be delivered in December 2024. This monitoring is divided into two review categories, in-depth and simplified. Simplified review requires accessibility testing using only automated accessibility evaluation tools, where as in-depth review requires both tools and manual review by accessibility specialists. 

Based on the public sector directive, the EU Commission has created a similar directive affecting some private sector bodies. The European Accessibility Act (EAA) was adopted by the EU in 2019 to complement the 2016 directive and UN CRPD \citep{eudirective2019}. The EAA will require medium and large sized private sector companies in the field of banking, travel and e-commerce starting to follow accessibility guidelines from June 2025 forward.


\chapter{Web accessibility evaluation\label{accessibility_evaluation}}

As web services undergo rapid development and continuous deployment, a continuous accessibility evaluation is crucial, mandated by both legislative organizations and end-users. The primary goal of assessing web accessibility is to promote digital inclusion by identifying and eliminating barriers, thereby expanding access to a wider audience. Multiple accessibility evaluation tools are available to help web content developers in ensuring inclusivity and conformance to accessibility standards.

\section{Methodologies}

Web accessibility evaluation is a process of assessing and determining to which extent a website is accessible to people with disabilities \citep[Chapter~26.2]{webaccessibility}. The evaluation of web accessibility involves analyzing each web page on the site against established standards and guidelines. The process of web accessibility evaluation can be categorized into three categories: automated testing, manual inspection, and user testing.

Automated testing is done by accessibility evaluation tools that are programmed to automatically parse and evaluate the source code of a web page based on guidelines \citep[Chapter~26.2]{webaccessibility}. Adopting automated testing early ensures that potential accessibility issues are identified and addressed with immediate feedback during the development cycle. However, automated testing tools can only cover those guidelines that are assessable through machine-based analysis.

Manual inspection is required to ensure that a web page conforms to accessibility standards \citep[Chapter~26.2]{webaccessibility}. A conformance review is the most frequently used methodology that involves evaluators checking if the web page meets criteria based on a checklist. \textcite{comparative_accessibility_methods} introduced and compared his Barrier Walkthrough methodology where evaluation is done in a more systematic way taking into consideration the context of the scenario and goal. An example of the scenario and goal is a person with vision impairment, using a specific assistive technology trying to fill in a form. Results show that the Barrier Walkthrough method is better at finding critical problems more accurately, but fails in finding all the accessibility problems \citep{comparative_accessibility_methods}. Nevertheless, a manual inspection conducted by different evaluators may result in different outcomes regardless of the methodology used \citep{accessibility_evaluation_experts, 10.1145/1878803.1878813_testability_expertise}. 

User testing is based on empirical usability testing. User testing is the most reliable accessibility evaluation method involving actual users with disabilities performing tasks on a web page \citep[Chapter~26.2]{webaccessibility}. Reliability is the consistency of the outcome regardless of whom performed the test. However, user testing is slow and costly. In addition, it is complex to set up the testing session and to take into account a diverse range of users \citep{comparative_accessibility_methods}

Multiple evaluation methodologies have been proposed combining the three categories described above \citep[Chapter~26.2.1]{webaccessibility}. It is recommended to always use more than one evaluation tool, as the technical implementation of each tool might differ ending up in different results. Manual inspection and user testing should always be done to ensure conformance. However, there is no consensus on how to combine these methods for a comprehensive accessibility study \citep[Chapter~26.2.1]{webaccessibility}.

\section{Automated tools\label{automated_tools}}

Automated Accessibility Evaluation Tools (AET) should be used to check if a web page conforms to accessibility guidelines. An accessibility evaluation tool parses the HTML and CSS code of a web page and checks that the web page conforms to the accessibility standards. These tools are helpful for quick accessibility evaluation. However, human evaluation is always required to check for full conformance as these tools might produce misleading results \citep{wcagevaluationtools}. Therefore, accessibility evaluation tools should be used as assistive evaluators when evaluating the conformance of a website.

There are dozens of tools to choose from, differing in functionality, coverage, and features \citep{tool_list}. Studies show that the amount of errors found on a page varies largely between tools \citep{comparison_10.1145/3371300.3383346, comparison_10.1145/3607720.3607722, tool_analysis_directive}. Where one tool can find a dozen errors, another tool can find thousands on the same page. \textcite{tool_analysis_directive} reviewed the monitoring reports from member states and did a comparative analysis on 10 free-to-use tools. Results show that almost all tools link found problems to WCAG success criteria. However, tools differed in the amount of criteria they checked and in the final representation of the coverage report. Furthermore, the transparency of tools on which success criteria they cover is not clear to users \citep{tool_analysis_directive}.

A method to compare and measure tool effectiveness was proposed by \textcite{Brajnik2004}. The effectiveness of a tool can be measured with correctness, completeness, and specificity. Correctness measures how tools report non-existing problems (false positives). Completeness measures how tools fail to find problems that exist (false negatives). A tool can be considered highly specific if it can detect a wide range of distinct accessibility issues, offering detailed insights into each problem. \textcite{benchmark_aet} studied the coverage, completeness, and correctness of six popular tools showing that a higher amount of accessibility issues on a page gives a higher completeness score, and in contradiction, on highly accessible pages the completeness score drops. In addition, the study shows that tools with high completeness scores report more easily false positives, reducing the correctness \citep{benchmark_aet}. \textcite{tooltransparency} evaluated four tools on their specificity and transparency. Results show that three of the tools share what success criteria and techniques they cover. However, differences were reported on how tools display the result to the user.

To mitigate the uncertainty and instability of accessibility evaluation tools the W3C has created a task force working on Accessibility Conformance Testing (ACT) rules \citep{act_overview}. The purpose of ACT rules is to standardize the interpretation of WCAG documents by creating technology-specific test cases that accessibility tool developers and accessibility methodology developers can use to test their outcomes. For unified reporting, the W3C has developed a resource description framework named Evaluation and Report Language (EARL). EARL is used to report conformance to the ACT rules. In addition, it can be used to combine results from different tools. However, currently, there is no standard on how to report the output of an automatic evaluation tool, and current tools differ in the reporting style \citep{tool_analysis_directive}.

\section{Tool coverage\label{coverage}}

Automated Accessibility Evaluation Tools (AET's) cover only around 20--30\% of all success criteria in the WCAG \citep{govukaccessibilityresults, dequecoverage, webaimmillions}. In a study conducted by \textcite{govukaccessibilityresults} they created a page with 142 accessibility issues and analyzed this page with 12 different tools (Nu HTML Checker excluded, not an AET). The lowest score was 17 \% by Google Accessibility Developer Tool and the highest scoring tool was SortSite with 40 \%. In addition, the coverage of all tools combined managed to find only 100 out of the 142 accessibility barriers, which supports the need to use multiple tools together when evaluating accessibility \citep{govukaccessibilityresults_blog}. \textcite{comparison_10.1145/3371300.3383346} conducted similar research where they found out that evaluating a page with multiple tools raises the coverage by 10--40 \%. However, \textcite{comparison_10.1145/3371300.3383346} did not study how to combine and remove duplicates from the results between each tool. One large developer and vendor of accessibility testing, Deque Systems, believes that the apprehension on the coverage should be based on real findings rather than which success criteria are covered \citep{dequecoverage}. 

Deque states that up to 57 \% of accessibility issues can be found when considering that there are usually multiple violations for one success criterion on a page \citep{dequecoverage}. From over 2000 in-depth accessibility evaluations they showed that in most cases automation finds more issues than manual review. Data in this study is based on A and AA levels of accessibility when WCAG 2.1 was the recommendation by W3C. The obsolete success criterion 4.1.1 Parsing in WCAG 2.2 was one of the most detected by automation with a proportion of 90.28 \% being found automatically. The six types of issues encountered with a significantly high proportion (percentage in brackets) found by automation are the following: 

\begin{itemize}
  \item 3.1.1 Language of page (91.81 \%, 1 995 issues)
  \item 1.4.3 Contrast (Minimum) (83.11 \%, 73 733 issues)
  \item 2.4.1 Bypass blocks (79 \%, 2 001 issues)
  \item 1.1.1 Non-text context (67.57 \%, 16 014 issues)
  \item 4.1.2 Name, role, value (54.42 \%, 26 276 issues)
  \item 1.3.1 Info and relationship (45.17 \%, 16 432 issues)
\end{itemize}

These six success criteria account for up to 52 \% of all accessibility issues found by automation. By removing the obsolete success criterion 4.1.1 from the data the total amount of found issues by automation is 53 \%. However, this study does not account for the six newly added success criteria in WCAG 2.2 Level AA. Moreover, \textcite{dequeaxe4_5} writes in Deque's blog that the only identified success criterion to be testable in WCAG 2.2 according to their promise, that is not returning false positives, is the success criterion 2.5.8 Target size \citep{dequeaxe4_5}. Deque's target is to ensure that the axe-core engine reports zero false positives to ensure that the results can be trusted by developers and accessibility evaluators \citep{dequecoverage}. Axe-core is an open-source accessibility testing engine for web browsers. It is used by millions of Github projects and it also works as the core for Deque's tools and Google Lighthouse. 

As WCAG version 2.2 is the new standard, according to the coverage report by \textcite{dequecoverage} there would be 17 out of 86 success criteria that automation can discover with certainty. An automated accessibility evaluator can determine that a page has a title or an image has an alternative text from the web page source code. However, these tools can not determine if a page title or alternative text for non-text content is descriptive \citep{wcag_checklist}.

\section{Manual evaluation}

Automated accessibility evaluation tools are the first step in finding accessibility problems on a web page. To increase the coverage, manual inspection is required. Manual evaluation is a costly and time-consuming process as each web page on the whole site has to be evaluated separately. Most of the success criteria in the WCAG can only be determined properly by human evaluation, such as criteria related to context. For example, test automation can detect if the alternative (alt) attribute is set for image tags in the HTML code. However, it can not determine if the alternative text describes the image correctly to users \citep{comparison_10.1145/3371300.3383346}. Additionally, expertise does matter when evaluating a page for conformance \citep{10.1145/1878803.1878813_testability_expertise, comparative_accessibility_methods}. In comparison to a novice, an expert in the field of web accessibility is more capable of finding accessibility barriers on a web page.

As there is no standard methodology for manual evaluation, each evaluator has a toolset that they use to evaluate individual pages manually. Using a semi-automated test tool is a great way to increase coverage of found barriers and to guide the evaluator in particular with wizard-based steps guiding the manual evaluation. Semi-automated test tools are used to evaluate a single page in a specific state for accessibility barriers. These tools run automated accessibility evaluations combined with, for example, user input wizards to cover more possible manually evaluated accessibility barriers. 

Two large accessibility evaluation consultant companies, Deque and Siteimprove, have their own paid semi-automated accessibility evaluation tools. \textcite{dequecoverage_semi} extended the same in-depth coverage study mentioned in Section \ref{coverage} on their semi-automated accessibility testing tool. They discovered that the coverage of found accessibility barriers is increased by 23\% when using their wizard-based semi-automated accessibility evaluation tool \citep{dequecoverage_semi}. The guided testing is incorporated into their test automation to help evaluators with a more systematic conformance evaluation. For example, when trying out Deque's tool free version subscription, the guided testing prompts to check if an alternative text for an image or page title is descriptive in a wizard to the evaluator. The page title wizard was the following question: \blockquote{The page title is 'Understanding WCAG 2.2 | WAI | W3C'. Does it accurately describe the purpose of the page?}. This question leaves the decision up to the evaluator to answer. To answer the question reliably, the evaluator needs to read the content of the page and form an understanding of the entire page.

\section{Use of AI}

With the emergence of multiple Generative AI tools for different contexts, such as music, image, or text generation, the use of AI and machine learning models in web accessibility has sparked discussion and gained focus in the field of accessibility. Companies within the accessibility evaluation industry, such as AudioEye and Deque, have incorporated AI features in their tools \citep{deque_igt, boia_improve_accessibility}. In addition, the current investment in AI has started a cautious discussion within the W3C WCAG Working Group where they are following the progress and possible use-cases of AI in web accessibility \citep{ai_wcag_email}. 

One of the most commonly found issues is a missing descriptive alternative text for images \citep{webaimmillions, dequecoverage}. Therefore, the potential of AI in generating alternative text for images is an interesting topic with multiple viewpoints \citep{ai_wcag_email, boia_alt_text, potential_for_ai}. Mozilla is experimenting with a machine learning model that runs locally on the user's computer to generate missing image alternatives when viewing PDF files within the browser \citep{alt_image_mozilla}. Even though there are tools to generate automatic alternative descriptions to images, the descriptions are not always considered appropriate as they do not take into account the context around the image \citep{accessibility_and_ai, boia_alt_text}. Besides, within the accessibility community, some say that AI should work as an assistant that should not make automatic decisions regarding accessibility \citep{ai_wcag_email, accessibility_and_ai}.

A case study conducted by \textcite{10.1145/3594806.3596542_case_study_gpt} showed that the use of ChatGPT in solving accessibility issues was able to provide correction to code and speed up the process of solving accessibility barriers. However, the study was conducted on two websites and the LLM had problems solving subjective issues, such as the color contrast between the background color and text color. \textcite{Lopez2024Turning} conducted an experiment where they tested if an LLM would be able to evaluate three different WCAG success criteria that require manual evaluation. In total, they had 39 ACT cases and a LLM was able to successfully evaluate 34 of them. However, they had to modify the prompt manually for more complex test cases to specify where to look for the correct information in the HTML tags.


\chapter{Methods\label{methods}}

Currently done and "how"


- Literature review chapter 4, get to know legislation from EU perspective, directives and memos in directive, read about guidelines -> EU Standard -> WCAG, explain accessibility and the accessibility guidelines in an understandable manner as short as possible with good literature (EU / State web pages)

- Literature review on accessibility evaluation, "accessibility evaluation tools" in google scholar, use ACM as main source, search "web accessibility" OR "accessibility evaluation" OR "evaluation of tools", filter by relevance and read papers, identify common themes and references, snowball method through recent papers to find the current themes and referred papers


\section{Copy from Preliminary Research Study Design}

Possible Research Questions

1. What is the current level of coverage of WCAG 2.2 success criteria by open-source accessibility evaluation tools?

2. What are the implications of the findings for web developers in terms of improving web accessibility and compliance with WCAG 2.2?

3. Which WCAG 2.2 success criteria can be effectively evaluated by these tools?

4. How can artificial intelligence be used to extend the evaluation of success criteria related to the context?

5. How does the performance of automated tools compare to manual evaluation in terms of accuracy, efficiency, and comprehensiveness in addressing WCAG 2.2 criteria?

6. How can the integration of Generative AI improve the functionality and effectiveness of existing open-source accessibility evaluation tools?

Research Goals

The primary goal of this research is to promote awareness of web accessibility and guide web developers in ensuring quality when developing websites using accessibility evaluation tools (AET). This involves assessing the coverage of AETs for success criteria and determining their accuracy. A secondary goal is to investigate the potential contributions of AI tools in

evaluating success criteria related to context. The focus is not on creating pass/fail metrics but on providing a human perspective in the evaluation process. The research explores the possibility of integrating Generative AI as a feedback loop in existing tools, adding a layer of "intelligence" that complements manual review for an enhanced accessibility evaluation toolset.

Research Methods

The research methodology includes a literature review on legislation, directives, and popular web accessibility evaluation tools from a web developer's standpoint. The study utilizes selected accessibility criteria to design an artifact for evaluating contextual accessibility on web pages, providing feedback to web developers when using accessibility evaluation tools.

Study Design

The study begins with a comprehensive background on the importance of web accessibility, its implications for web developers, and a review of accessibility evaluation tools based on existing legislation. The “How to meet WCAG” guidelines are filtered based on legislation requirements and tags such as "Developing", HTML, CSS, and ARIA technologies. A literature review assesses the coverage of open-source web accessibility evaluation tools in meeting web developers' compliance requirements. Investigation into the performance of Large Language Models (LLM) and image recognition tools on their ability to match for example alternative text for images (criteria 1.1.1) in HTML source code or page title criteria 2.4.2. The study's scope involves constructing a solution idea based on prompt engineering for a popular image recognition/LLM and assessing its performance through human perception using black-box testing.
\chapter{Results\label{results}}

This chapter presents the results of this thesis. 

\section{Limitations of web accessibility evaluation tools}

The literature review done in Chapter \ref{accessibility_evaluation} answers RQ1. In total there are 86 success criteria in the WCAG 2.2. Out of these 86 success criteria, test automation covers reliably (zero false positives) 17 the 86 success criteria in the WCAG 2.2. Using manual evaluation with semi-automated web accessibility evaluation tools increases the coverage and adds to the sufficiency of these 17 found by automation.

Research indicates that the WCAG 2.2 success criteria are hard to interpret correctly by accessibility evaluator tools developers. Papers on Automated AET's show that there is a significant distribution on the amount of found accessibility barriers on the same page. Studies conducted points out that even experts evaluating the same page can end up in different result. Additionally, the transparency of results by AET's to accessibility evaluators even further adds to the complexity of interpreting which success criteria has been sufficiently evaluated.

Web accessibility evaluation tools helps in conformance reviews by guiding the evaluator with wizards to further increase the coverage. However, the outcome of these wizards are subjective to the evaluator. Therefore, expertise matter when using semi-automated AET's.

\section{Capabilities of LLM's for Page Titled evaluation}

This section describes and evaluates the outcomes of the artifact. Results are evaluated based on usefulness, accuracy and consistency. Each iteration is analysed in an own subsection. For each iteration, the \textcite{act_rule_g88} ACT test cases are used. Each test case were executed in sequences of five using the ChatGPT user interface ensuring the ChatGPT 3.5, see Appendix \ref{appendix:iterations}.

\subsection{First iteration results}

In most of the test cases, ChatGPT was capable of evaluating if the title was descriptive based on the content of the web page. However, problems occurred in failing and inapplicable test cases. Passing test cases had the best outcome based on evaluation criteria, whereas failing and inapplicable test cases had more variances and evident mistakes.

\subsubsection{Passing test cases}

Most promising results can be seen in the passing test cases P1, P2 and P3. The output has a coherent structure for each test case. When assessing the output in detail, the LLM provides reasoning for each rule in the input. However, the P2 test case's fourth sequence outcome states that the code snippet has only one title element, even though the test case has two.

In the P3 test case, it can be noted that ChatGPT 3.5 is not capable of identifying correctly where the title element is present, and in three out of five sequences it falsefully states that the title element is within the head section of the HTML document, even though it is in the body section. However, this is a minor mistake, as most browsers are able to fix and set the title correctly.

The accuracy regarding the accessibility evaluation is described in a short paragraph at the end of the output for all the test cases. Each test case and its sequences have the same outcome from a contextual perspective, therefore the LLM is consistent when evaluating these passed test cases.

\subsubsection{Failing test cases}

Inconsistency and inaccuracy is found in the failing test cases F1 and F2. Additionally, there are irrational behavior in same output when it is assessing each rule individually. Furthermore, the length of the answers from the Generative AI are longer then in passed case. In comparison to passing test cases, where the output explicitly states conformance, in failing cases the output tends to provide partial conformance or ambiguity when interpreting the outcome. 

\subsubsection{F1 test case}

In the fifth sequence of F1, see \ref{F1-1}, the LLM explicitly indicates that the page conforms with the 2.4.2 Page Titled success criterion. In addition, the LLM seems to provide irrational conclusion for the second rule \textit{Check that the title is relevant to the content of the Web page} with an output of:

\begin{quote}
    The title of the web page is "Apple harvesting season". This title seems relevant to the content of the page, which discusses the harvest timing of clementines, a type of fruit.
\end{quote}

The first sequence of F1 does not explicitly mention if it conforms or not. A closer look on the answer indicates that the LLM does not understand that the whole context is provided, therefore uncertainty in accuracy is visible. However, the uncertainty is also reflected in the overall description asking for further evaluation. Therefore, the answer is relevant from the evaluators perspective who is using this GenAI accessibility assistant when evaluating web pages for conformance.

\subsubsection{F2 test case}

The F2 test case seems to be a major problem for the artifact when considering each of the evaluating criteria. Evaluating each output in detail, an irrational behavior is noticeable. The LLM checks that there is a title that is relevant to the content of the page before it should exclude the first title that is not descriptive in this test case according to the fourth rule. This indicates that the scope of the rule could be moved further up to make the four rules in a logical order. However, the test case input talks about \textit{first title element} and looking at the fifth sequence answer: \textit{The rule specifies that browsers generally recognize only the first title element, so it's assumed that "First title is incorrect" would not be considered by browsers, and "Clementine harvesting season" is the effective title.} would indicate that the test case could also be the problem for the LLM if it explicitly drops out the title based on the content, not order. Same characteristics in answer is found in all other except the third sequence.

Due to possible problems described above, the accuracy for this test case is not on a sufficient level. However, a closer look at the reasoning behind the rules shows that the usefulness is still valid, as the LLM points out that the second title would be more relevant.

\subsubsection{F3 test case}

Each answer provides useful suggestion on how to improve the title description and the answers are consistent through each sequence. All other sequences explicitly states partial conformance, scoring well in accuracy, however the chosen words for the first sequence \textit{"Overall, while the web page meets the basic requirement of having a title, it could improve its accessibility by making the title more descriptive and directly relevant to the content..."} is the only one that stands out from these answers. 

\subsubsection{Inapplicable test case (I1)}

The inapplicable test case was the most inaccurate and inconsistent for the LLM. In two sequences it was able to state that the web page lacks a proper title element for the web page, which is the correct outcome. On other test sequences it satisfied the success criterion. However, as this is a zero shot prompt there are no examples on how to determine applicability for the LLM even though it was able to in two of the sequences.


\subsection{Second iteration results}

The second iteration had a significant affect on the outcome. Improvements in accuracy are evident in failing test cases. With the removal of the fourth rule and using it's knowledge as a separate instruction in the artifact, the length of the output decreased, improving the efficiency of the evaluator.

\subsubsection{Passing test cases}

In regards of accuracy, all sequences within the passing test cases was evaluated correctly. Therefore, each test cases sequences were consistent in regards of accuracy. Additionally‚ with changes to the artifact, the P2 test case "Second title is ignored" is not considered in any of the sequences. 

As in the first iteration of the artifact, the LLM still had issues to identify in the P3 test case that the title element is not within the head section. However, as earlier stated, this is a minor flaw that does not affect the results.

\subsubsection{Failing test cases}
\chapter{Discussion\label{discussion}}

This chapter contains a discussion of the study, implications, limitations, and potential future research. Section 6.1 is a summary of the research questions. Section 6.2 a discussion on the study result and implications are presented. Section 6.3 presents threats to the validity of the study. Section 6.4 promotes potential future research.

\section{Summary of main findings}

Below is a recap of the research questions and a summary of the answers.

\begin{itemize}
    \item \textbf{RQ1.} What are the limitations of web accessibility evaluation tools in assessing compliance with Web Content Accessibility Guidelines (WCAG)?
\end{itemize}

    A multivocal literature review answered the current limitations of web accessibility evaluation tools. Test automation covers 17 out of the 86 success criteria in the WCAG 2.2. However, test automation tools are not capable of thorough evaluation that would meet the WCAG set standard for some of these 17 success criteria. 

    Sufficiently evaluating conformance requires an accessibility specialist evaluation which is a tedious process, as web pages are more complicated than ever. Semi-automated accessibility evaluation tools help evaluators by guiding them through the most common accessibility barriers found on web pages. However, evaluating for a success criterion requiring interpretation of content can end up in a different outcome depending on the evaluators' point of view.
    
\begin{itemize}
    \item \textbf{RQ2.} How does Generative AI assist in addressing these limitations?
\end{itemize}

    Generative AI can be utilized to address these limitations. Large Language Models can assist evaluators in conformance checks that require interpretation of content. On pages with longer content, the automatic context analysis by LLMs could potentially decrease the overall time for conformance reviews. However, as results show, with zero-shot chain of thought prompting, LLMs are not capable of reliably determining inapplicability.

\section{Study result analysis}

This study shows that even though legislation is moving forward in regards to accessibility, the nature of accessibility and accessibility evaluation is complex, and requires expertise from designers, developers, quality assurance, and accessibility reviewers. An accessibility specialist needs to have a thorough knowledge of the WCAG documents, as no accessibility evaluation tool has 100\% coverage. In addition, web developers, designers, and content creators ought to study the same WCAG documents. 

An accessibility evaluation tool developer has to understand the ACT rules and sufficient techniques used to check for conformance, as well as how browsers work, to develop a reliable and robust tool for accessibility evaluators. Transparency of evaluation tools helps the evaluator understand which success criteria it covers and to what extent. Results show that Large Language Models could be utilized to improve the sufficiency, and possibly help ease the conformance review process, when using semi-automated accessibility evaluation tools. 

\subsection{Prompt iteration}

These results build on existing evidence that LLMs are good Zero-Shot reasoners \citep{kojima2023large}. With rigorous prompt iteration, the accuracy and quality of the outcome improved. By evaluating the outcome of the LLM, patterns can be detected where the LLM fails to provide reasoning for checks it performed, giving possible directions for improvements. An example is the order of the conditional checks the LLM should take into account. Therefore, an imperative approach to how the LLM should operate step by step combined with the zero-shot chain of thought improved the quality of the outcome. However, as the second iteration of the artifact went through multiple. Therefore, uncertainty whether moving the fourth rule to be a manually created sentence, or if replacing the question with the zero-shot chain of thought tested by \textcite{kojima2023large}, improved the accuracy and consistency in the second iteration.

\section{Limitations}

The lack of a thorough evaluation, such as surveys or interviews with potential users of this artifact, is a concern of the validity of this study. The evaluation of usefulness is solely based on the empirical findings of the thesis writer. The artifact is a proof of concept and has not been evaluated by accessibility evaluators for usefulness in a accessibility conformance reviews. Therefore, a proof of suitability, evaluating whether the LLM would assist and speed up conformance reviews, is yet to be evaluated that would support the findings of the study. 

Two concerns regarding the chosen LLM are that this study was solely done using the ChatGPT 3.5 user interface due to it being free to use, and that the LLM is a closed-source tool. Therefore, between the iterations, there is no knowledge if there have been any improvements made by OpenAI to the language model. However, according to OpenAI, the ChatGPT 3.5 model was trained only with data available in early 2022 \citep{openai_35}. Therefore, it can be assumed that the test cases available at \textcite{act_rule_g88} should not be part of the LLM knowledge base. Additionally, the replication of this study is challenging, as no date-fixed version of the model was used. In addition, during September 2024 ChatGPT 3.5 became obsolete for free tier users using the user interface, and a new alternative ChatGPT 4o was announced.

Additionally, the characteristics of LLMs, such as non-deterministic output given the same input, or the limited amount of characters that you can input, are a threat to the external validity of the study. This study was conducted using very short code snippets, therefore no input limits were hit. Even though, in the second iteration all the passing and failing test case sequences were correctly evaluated by the LLM, this does not guarantee that the LLM would always correctly evaluate due to the characteristics of LLM being non-deterministic. However, a better design could reduce the input limitation threat. For example, in this study setting, parsing the HTML code, picking the first title element encountered, and the content within the HTML body or main tag would significantly reduce the number of characters sent to the LLM. In addition, the non-deterministic behavior would cause problems if your accessibility evaluation tool promises zero false positives when evaluating for accessibility.

It is beyond the scope of this study to evaluate how the LLM would work if the website language would be other than English or with possible other language models available either provided by someone or running the models locally.

\section{Future research}

Future studies should take into account language used on the website as accessibility barriers are not only limited to English. In addition, as this study was limited to the short ACT test cases, a case study where the artifact has been implemented into a semi-automatic accessibility evaluation tool with a feedback loop from the evaluator using the AI feature, would provide insight on how the LLM is evaluating larger websites with more content. 


\chapter{Conclusions\label{conclusions}}

The goal of this study was to find out how sufficiently current accessibility evaluation tools test the WCAG 2.2 success criteria and to assess the potential of Large Language Models in these tools during conformance evaluation. It is suggested to use multiple semi-automated and automated accessibility evaluation tools during conformance evaluation to improve the coverage of found accessibility barriers. The findings show that, when given conditions to check for in the prompt, LLMs can assist in that the HTML code has a title, the title is relevant to the page content, and the title identifies the page. However, a human should evaluate the correctness of the output. Additionally, an accessibility conformance evaluator does not necessarily need to be a subject matter expert regarding website content, as an LLM could help evaluate the context based criteria. Integrating Generative AI, such as LLMs, into AETs could enhance the accuracy and efficiency of conformance evaluation, enabling accessibility reviewers to carry out more comprehensive accessibility assessments more easily.




%%%%%%%%%%%%%%%%%%%%%%%%%%%%%%%%%%%%%%%%%%%%%%%%%%%%%%%%%
%\cleardoublepage                          %fixes the position of bibliography in bookmarks
%\phantomsection
\addcontentsline{toc}{chapter}{\bibname}  % This lines adds the bibliography to the ToC
\printbibliography

%%%%%%%%%%%%%%%%%%%%%%%%%%%%%%%%%%%%%%%%%%%%%%%%%%%%%%%%%
\backmatter
\begin{appendices}

%% A sample Appendix

\appendix{Artifact test case output shared links\label{appendix:iterations}}

The conversation is renamed in form of P1-1-1, where P1 refers to the test case in the \textcite{act_rule_g88}. First number describes the iteration, that is, which version of the artifact and the last number the sequence number ran for the iteration.

\section{Iteration 1}

\subsection{P1}

\begin{enumerate}
    \item https://chat.openai.com/share/5589d0f3-a3e3-4aee-bd09-63fb87503f4e
    \item https://chat.openai.com/share/c9cbb56f-9da3-42f4-928c-7931e1f6d6c6
    \item https://chat.openai.com/share/73173b78-220d-49de-85fd-a659f38ce09d
    \item https://chat.openai.com/share/b262027f-b852-4ffb-8691-bbff11a7b660
    \item https://chat.openai.com/share/e0c1b60f-b309-4996-a17b-21a3020ee029
\end{enumerate}

\subsection{P2}

\begin{enumerate}
    \item https://chat.openai.com/share/fa9ad3e2-ad93-4429-b6ee-38a7ef6bad0e
    \item https://chat.openai.com/share/79e7dd08-a79e-4594-80ef-87b7a07b64b2
    \item https://chat.openai.com/share/aa354068-bc22-4fdf-9ac3-4a783be9b5d6
    \item https://chat.openai.com/share/fb200c6f-a83d-4363-81e0-3e869ed52f48
    \item https://chat.openai.com/share/af4efc55-de9b-4b0b-b742-074798a1ba0c
\end{enumerate}

\subsection{P3}

\begin{enumerate}
    \item https://chat.openai.com/share/d2af5c6b-9489-4836-998f-c8af4d99213c
    \item https://chat.openai.com/share/a7e9bb5a-20ea-4eac-820c-dede05d34279
    \item https://chat.openai.com/share/18227ab9-65a6-4eaa-868e-b50774e96199
    \item https://chat.openai.com/share/dabac1ef-8ca3-4765-9366-e9296f9ddd7d
    \item https://chat.openai.com/share/9dac3e30-e2f0-4ae5-9f9a-20e3d7775d6c
\end{enumerate}

\subsection{F1}

\begin{enumerate}
    \item https://chat.openai.com/share/87a9d162-b402-40fc-8bfa-664c6a5ba3a2
    \item https://chat.openai.com/share/34809a27-8665-4f6c-84ab-7eb5b243f065
    \item https://chat.openai.com/share/ccb471b1-522e-4b85-9fec-b405e823086b
    \item https://chat.openai.com/share/af1997ef-ef04-46fc-8cfe-a1f8bba6daab
    \item https://chat.openai.com/share/c571e9ed-b9ab-4e54-9389-67c456776708
\end{enumerate}

\subsection{F2}

\begin{enumerate}
    \item https://chat.openai.com/share/866af13f-2653-4d60-99e5-234f67a2be93
    \item https://chat.openai.com/share/1da36f2b-f112-4881-89ad-af37c4ee6aaf
    \item https://chat.openai.com/share/e33b7b58-a43c-4c0b-be08-4e05d31af6f2
    \item https://chat.openai.com/share/acec840d-8e8b-4a84-8499-4157e62efc41
    \item https://chat.openai.com/share/4bc77120-396e-43a3-8292-8594259b331b
\end{enumerate}

\subsection{F3}

\begin{enumerate}
    \item https://chat.openai.com/share/86f2fb56-fd98-4791-a764-0d61b13c920f
    \item https://chat.openai.com/share/f1814074-d83d-429e-ae03-1c6843d4aac0
    \item https://chat.openai.com/share/08684ef8-34e2-4b72-9d00-5215b0d90e69
    \item https://chat.openai.com/share/8cb6a180-8ff0-48e2-aa3c-d5f73f3993e6
    \item https://chat.openai.com/share/7e2f1b2a-a6c7-45f4-9901-7f3cba63c24e
\end{enumerate}

\subsection{I1}

\begin{enumerate}
    \item https://chat.openai.com/share/d6c95007-b965-4ca9-b176-4e3f9899bb0d
    \item https://chat.openai.com/share/a2789f63-0ad6-4e56-9a3f-0fc2de56fdae
    \item https://chat.openai.com/share/63d5dc58-6c52-4055-9357-b6e97fbbeeb0
    \item https://chat.openai.com/share/20b42730-76de-4345-a6db-ef130d43b1a9
    \item https://chat.openai.com/share/9ddfd254-5917-430f-8984-8610e6608c70
\end{enumerate}
%% another appendix
%% \include{instructions/instructions_english}
%% yet another appendix
%% \include{instructions/instructions_finnish}

% BSc instructions
%\include{instructions/bsc_finnish_contents}
%\include{instructions/bsc_english_contents}


\end{appendices}
%%%%%%%%%%%%%%%%%%%%%%%%%%%%%%%%%%%%%%%%%%%%%%%%%%%%%%%%%

\end{document}
