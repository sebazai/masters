\chapter{Conclusions\label{conclusions}}

The goal of this study was to assess how many of the WCAG 2.2 success criteria current state-of-the-art Accessibility Evaluation Tools test and to assess the potential of Large Language Models in these tools during conformance reviews. The findings shows that, when given conditions to check for in the prompt, LLMs are capable of verifying that the HTML code has a title, that the title is relevant to the page content, and the title identifies the page. This suggests that an accessibility conformance reviewer does not necessarily need to be a subject matter expert regarding website content, as a LLM can help in evaluating these aspects. Integrating LLMs into AETs could enhance the accuracy and efficiency of conformance reviews, enabling reviewers to easier carry out more comprehensive accessibility assessments.


