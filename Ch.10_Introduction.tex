\chapter{Introduction\label{intro}}

Within the past decade, web accessibility legislation in the EU has enforced the public sector websites to comply with web accessibility guidelines \citep{eudirective2016}. Additionally, the European Accessibility Act will force some private sector websites to comply after the second half of 2025 \citep{eudirective2019}. The legislation aims to ensure equal access to digital information and services for people with disabilities. However, more than 95\% of the top million websites have accessibility issues that are detectable with Accessibility Evaluation Tools (AETs) \citep{webaimmillions}.

There are multiple AETs to help assess accessibility compliance \citep{govukaccessibility}. However, these tools cover reliably around 20--30\% of the 86 success criteria \citep{govukaccessibilityresults, dequecoverage, webaimmillions}. Therefore, human evaluation and expertise in accessibility are necessary to uncover barriers that affect people with disabilities \citep{wcagevaluationtools}. An accessibility evaluation tool can detect if a website has a title, but a human has to evaluate if the title describes the content of the page. 

This thesis explores web accessibility evaluation and the potential of Large Language Models (LLMs) in web accessibility evaluation, emphasizing 2.4.2 Page Titled Success Criterion that requires human evaluation. A design science research approach is taken to build an input prompt for an LLM to help evaluate if the title describes the content. The thesis is structured as follows.

Chapter 2 introduction to web accessibility, covering definitions, common disabilities affecting web use, implementation, guidelines, and legislation in the EU. Chapter 3 focuses on accessibility evaluation methodologies, accessibility evaluation tools, manual evaluation, and the potential role of AI within accessibility. Evaluation methodologies are presented, following the capabilities of AETs, detailing which criteria they cover, how the effectiveness and reliability of AETs have been studied, and what has been done to standardize the AET's development.

Chapter 4 details the selected research methods, research questions, problem identification, selected tools and scope, and the two iteration phases of the input prompt created in the study, along with the evaluation metrics used. The results of each iteration are presented in Chapter 5, highlighting the accuracy, consistency, and perceived usefulness of the LLM's output.

Chapter 6 summarizes the main findings, discusses the implications, and addresses study limitations. The thesis concludes by emphasizing the importance of web accessibility evaluation through AI-assisted approaches.