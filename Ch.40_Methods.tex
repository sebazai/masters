\chapter{Research methods\label{methods}}

This thesis will use a qualitative method based on an iterative manner of the Design Science methodology \citep{designsciencemethodology, iterativedesignscience}. The main goal of this research is do a literature review to identify a problem in a specific context and attempt to create an artifact as a solution to found problem. The context of this thesis will be to research web accessibility evaluation tools and to identify how Generative AI could assist accessibility evaluators to resolve barriers on the web. 

<Insert figure here from \cite{iterativedesignscience}>

The research questions for the thesis are following:

\begin{itemize}
    \item \textbf{RQ1.} What are the limitations of web accessibility evaluation tools in assessing compliance with Web Content Accessibility Guidelines (WCAG)?
    \item \textbf{RQ2.} How well does Generative AI assist to address these limitations?
    \item \textbf{RQ3.} How could these features be incorporated in existing accessibility evaluation tools?
\end{itemize}

To identify current limitations for RQ1 a thorough literature review was conducted with focus on web accessibility from a legislative perspective in the EU.  Chapter \ref{accessibility} describes the progress done to ensure an accessible web. Based on findings from chapter \ref{accessibility} the following search string was used in the ACM Digital Library: "accessibility evaluation" OR "evaluation of tools" OR "comparing tools". Filters applied were: Research Article and Publication Date between 2019 - 2024 to find currently relevant publication within the field. A total of 150 research articles were found. Found papers were selected based on their title and abstract. An inclusion criteria was used based on if either title or abstract mentioned comparing tools, the EU Directive or the WCAG guidelines. Exclusion criteria was used if the paper title or abstract had mentioned one of the following: specific disability, specific technology, mobile accessibility or the study was conducted in a country. A snowballing method was used to identify the most commonly referred papers which are also used in this thesis as references. Additionally, found researches were searched for to find relevant literature.






- Literature review chapter 4, get to know legislation from EU perspective, directives and memos in directive, read about guidelines -> EU Standard -> WCAG, explain accessibility and the accessibility guidelines in an understandable manner as short as possible with good literature (EU / State web pages)

- Literature review on accessibility evaluation, "accessibility evaluation tools" in google scholar, use ACM as main source, search "web accessibility" OR "accessibility evaluation" OR "evaluation of tools", filter by relevance and read papers, identify common themes and references, snowball method through recent papers to find the current themes and referred papers
