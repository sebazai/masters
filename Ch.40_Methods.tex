\chapter{Methods\label{methods}}

Currently done and "how"


- Literature review chapter 4, get to know legislation from EU perspective, directives and memos in directive, read about guidelines -> EU Standard -> WCAG, explain accessibility and the accessibility guidelines in an understandable manner as short as possible with good literature (EU / State web pages)

- Literature review on accessibility evaluation, "accessibility evaluation tools" in google scholar, use ACM as main source, search "web accessibility" OR "accessibility evaluation" OR "evaluation of tools", filter by relevance and read papers, identify common themes and references, snowball method through recent papers to find the current themes and referred papers


\section{Copy from Preliminary Research Study Design}

Possible Research Questions

1. What is the current level of coverage of WCAG 2.2 success criteria by open-source accessibility evaluation tools?

2. What are the implications of the findings for web developers in terms of improving web accessibility and compliance with WCAG 2.2?

3. Which WCAG 2.2 success criteria can be effectively evaluated by these tools?

4. How can artificial intelligence be used to extend the evaluation of success criteria related to the context?

5. How does the performance of automated tools compare to manual evaluation in terms of accuracy, efficiency, and comprehensiveness in addressing WCAG 2.2 criteria?

6. How can the integration of Generative AI improve the functionality and effectiveness of existing open-source accessibility evaluation tools?

Research Goals

The primary goal of this research is to promote awareness of web accessibility and guide web developers in ensuring quality when developing websites using accessibility evaluation tools (AET). This involves assessing the coverage of AETs for success criteria and determining their accuracy. A secondary goal is to investigate the potential contributions of AI tools in

evaluating success criteria related to context. The focus is not on creating pass/fail metrics but on providing a human perspective in the evaluation process. The research explores the possibility of integrating Generative AI as a feedback loop in existing tools, adding a layer of "intelligence" that complements manual review for an enhanced accessibility evaluation toolset.

Research Methods

The research methodology includes a literature review on legislation, directives, and popular web accessibility evaluation tools from a web developer's standpoint. The study utilizes selected accessibility criteria to design an artifact for evaluating contextual accessibility on web pages, providing feedback to web developers when using accessibility evaluation tools.

Study Design

The study begins with a comprehensive background on the importance of web accessibility, its implications for web developers, and a review of accessibility evaluation tools based on existing legislation. The “How to meet WCAG” guidelines are filtered based on legislation requirements and tags such as "Developing", HTML, CSS, and ARIA technologies. A literature review assesses the coverage of open-source web accessibility evaluation tools in meeting web developers' compliance requirements. Investigation into the performance of Large Language Models (LLM) and image recognition tools on their ability to match for example alternative text for images (criteria 1.1.1) in HTML source code or page title criteria 2.4.2. The study's scope involves constructing a solution idea based on prompt engineering for a popular image recognition/LLM and assessing its performance through human perception using black-box testing.