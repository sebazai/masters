\chapter{Web accessibility evaluation\label{accessibility_evaluation}}

As web services undergo rapid development and continuous deployment, a continuous accessibility evaluation is crucial, mandated by both legislative organizations and end-users. The primary goal of assessing web accessibility is to promote digital inclusion by identifying and eliminating barriers, thereby expanding access to a wider audience. Multiple accessibility evaluation tools are available to help web content developers in ensuring inclusivity and conformance to accessibility standards.

\section{Methodologies}

Web accessibility evaluation is a process of assessing and determining to which extent a website is accessible to people with disabilities \citep[Chapter~26.2]{webaccessibility}. The evaluation of web accessibility involves analyzing a website against established standards and guidelines. The process of web accessibility evaluation can be categorized into three categorize: automated testing, manual inspection and user testing.

Automated testing is done by accessibility evaluation tools that are programmed to automatically parse and evaluate the source code of a web page based on guidelines \citep[Chapter~26.2]{webaccessibility}. Adopting automated testing early ensures that potential accessibility issues are identified and addressed with immediate feedback during the development cycle. However, automated testing tools can only cover those guidelines that are assessable through machine-based analysis.

Manual inspection is required to ensure that a web page conforms to accessibility standards \citep[Chapter~26.2]{webaccessibility}. Conformance review is the most used methodology that involves evaluators checking if the website meets criteria based on a checklist. \textcite{comparative_accessibility_methods} introduced and compared his Barrier Walkthrough methodology where evaluation is done in a more systematic way taking into consider the context of the scenario and goal. An example of the scenario and goal is a person with vision impairment, using a specific assistive technology trying to fill in a form. Results show that the Barrier Walkthrough method is better at finding critical problems more accurately, but fails in finding all the accessibility problems \citep{comparative_accessibility_methods}. Nevertheless, manual inspection conducted by different evaluators may result in different outcome regardless of methodology used \citep{accessibility_evaluation_experts, 10.1145/1878803.1878813_testability_expertise}. 

User testing is based on empirical usability testing. User testing is the most reliable accessibility evaluation method involving actual users with disabilities performing tasks on a web page \citep[Chapter~26.2]{webaccessibility}. Reliability is the consistency of the outcome regardless of whom performed the test. However, user testing is slow and costly. In addition, it is complex to set up the testing session and to take into account a diverse range of users \citep{comparative_accessibility_methods}

Multiple evaluation methodologies have been proposed combining the three categories described above \citep[Chapter~26.2.1]{webaccessibility}. It is recommended to always use more then one evaluation tool, as the technical implementation of each tool might differ ending up in different results. Manual inspection and user testing should always be done to ensure conformance. However, there is no consensus on how to combine these methods for a comprehensive accessibility study \citep[Chapter~26.2.1]{webaccessibility}.

\section{Automated tools\label{automated_tools}}

Automated Accessibility Evaluation Tools (AET) should be used to check if a web page conforms to accessibility guidelines. An accessibility evaluation tool parses the HTML and CSS code of a web page and checks that the web page conforms to the accessibility standards. These tools are helpful for quick accessibility evaluation. However, human evaluation is always required to check for full conformance as these tools might produce misleading results \citep{wcagevaluationtools}. Therefore, accessibility evaluation tools should be used as assistive evaluators when examining the conformance of a web page.

There are dozens of tools to choose from, differing in functionality, coverage and features \citep{tool_list}. Studies show that the amount of errors found on a page varies largely between tools \citep{comparison_10.1145/3371300.3383346, comparison_10.1145/3607720.3607722, tool_analysis_directive}. Where one can find dozens of errors another tool can find thousand on the same page. \textcite{tool_analysis_directive} reviewed the monitoring reports from member states and did a comparative analysis on 10 free to use tools. Results show that almost all tools link found problems to WCAG success criteria. However, tools differed in the amount of criteria they check and in the final representation of the coverage report. Furthermore, the transparency of tools on which success criteria they cover is not clear to users \citep{tool_analysis_directive}.

A method to compare and measure tool effectiveness was proposed by \textcite{Brajnik2004}. Effectiveness of a tool can be measured with correctness, completeness and specificity. Correctness measures how tools report non-existing problems (false positives). Completeness measures how tools fail to find problems that exist (false negative). A tool can be considered highly specific if it can detect a wide range of distinct accessibility issues, offering detailed insights into each problem. \textcite{benchmark_aet} studied the coverage, completeness and correctness of six popular tools showing that higher amount of accessibility issues on a page gives a higher completeness score and in contradiction, on highly accessible pages the completeness score drops. In addition, the study shows that tools with high completeness score reports more easily false positives, reducing the correctness \citep{benchmark_aet}. \textcite{tooltransparency} evaluated four tools on their specificity and transparency. Results show that three of the tools share what success criteria and techniques they cover. However, differences were reported on how tools display the end-result to the user.

To mitigate the uncertainty and instability of accessibility evaluation tools the W3C has created a task force working on Accessibility Conformance Testing (ACT) rules \citep{act_overview}. The purpose of ACT rules is to standardize the interpretation of WCAG documents by creating technology specific test cases that accessibility tool developers and accessibility methodology developers can use to test their outcome. For unified reporting, the W3C has developed a resource description framework named Evaluation and Report Language (EARL). EARL is used to report conformance of the ACT rules. In addition, it can be used to combine results from different tools. However, currently there is no standard on how to report the outcome of an automatic evaluation and current tools differ in the reporting \citep{tool_analysis_directive}.

\section{Coverage\label{coverage}}

Automated accessibility evaluation tools (AET) cover only around 20--30 \% of all success criteria in the WCAG \citep{govukaccessibilityresults, webaimmillions, dequecoverage}. In a study conducted by \textcite{govukaccessibilityresults} they created a page with 142 accessibility issues and analyzed this page with 12 different tools (Nu HTML Checker excluded, not an AET). The lowest score was 17 \% by Google Accessibility Developer Tool and highest scoring tool was SortSite with 40 \%. In addition, the coverage of all tools combined managed to find only 100 out of the 142 accessibility barriers, which supports the need to use multiple tools together when evaluating accessibility \citep{govukaccessibilityresults_blog}. \textcite{comparison_10.1145/3371300.3383346} conducted a similar research where they found out that evaluating a page with multiple tools raises the coverage by 10--40 \%. However, \textcite{comparison_10.1145/3371300.3383346} did not study how to combine and remove duplicates from the results between each tool. One large developer and vendor of accessibility testing, Deque Systems, believes that the apprehension on the coverage should be based on real findings rather than which success criteria are covered \citep{dequecoverage}. 

Deque states that up to 57 \% of accessibility issues can be found when considering that there are usually multiple violations for one success criterion on a page \citep{dequecoverage}. From over 2000 in-depth accessibility evaluations they showed that in most cases automation finds more issues then manual review. Data in this study is based on A and AA level of accessibility when WCAG 2.1 was the recommendation from W3C. The obsolete success criterion 4.1.1 Parsing in WCAG 2.2 was one of the most detected by automation with a proportion of 90.28 \% being found automatically. The six type of issues encountered with a significantly high proportion found by automation are the following: 

\begin{itemize}
  \item 3.1.1 Language of page (91.81 \%, 1 995 issues)
  \item 1.4.3 Contrast (Minimum) (83.11 \%, 73 733 issues)
  \item 2.4.1 Bypass blocks (79 \%, 2 001 issues)
  \item 1.1.1 Non-text context (67.57 \%, 16 014 issues)
  \item 4.1.2 Name, role, value (54.42 \%, 26 276 issues)
  \item 1.3.1 Info and relationship (45.17 \%, 16 432 issues)
\end{itemize}

These six success criteria accounts for up to 52 \% of all accessibility issues found by automation. By removing the obsolete success criterion 4.1.1 from the data the total amount of found issues by automation is 53 \%. However, this study does not account for the six newly added success criteria in WCAG 2.2 Level AA. Moreover, Deque writes that the only identified success criterion to be testable and not returning false positives in WCAG 2.2 is the success crtierion 2.5.8 Target size \citep{dequeaxe4_5}. Deque's target is to ensure that the axe-core engine reports zero false positives to ensure that the results can be trusted \citep{dequecoverage}. Axe-core is an open-source accessibility testing engine for web browsers. It is used by millions of Github projects and it also works as the core for Deques own tools, and for Google Lighthouse. 

As WCAG version 2.2 being the new standard, according to the coverage report by \textcite{dequecoverage} there would be only 17 of out 86 success criteria that automation can discover correctly. An automated accessibility evaluator can determine that a page has a title or an image has an alternative text from the web page source code. However, these tools can not determine if the title or alternative text is descriptive that is a requirement to fulfill success criteria 1.1.1 Non-text content for images or 2.4.2 Page titled \citep{wcag_checklist}.

\section{Manual evaluation}

Automated accessibility evaluation tools is the first step on finding accessibility problems on a web page. To increase the coverage, manual inspection is required. Manual evaluation is a costly and time consuming process. Most of the success criteria in the WCAG can only be determined properly by human evaluation, such as criteria related to context. For example, test automation can detect if the alternative (alt) attribute is set for image tags in the HTML code. However, it can not determine if the alternative text describes the image correctly to users \citep{comparison_10.1145/3371300.3383346}. Additionally, expertise does matter when evaluating a page for conformance \citep{10.1145/1878803.1878813_testability_expertise, comparative_accessibility_methods}.

As there is no standard methodology for manual evaluation, each evaluator has their own tool set that they use to evaluate individual pages manually. Using semi-automated test tool is a great way to increase coverage of found barriers and to guide the evaluator. Semi-automated test tools are used to evaluate a single page in a specific state for accessibility barriers. These tools run automated accessibility evaluation combined with user input wizards to cover more of possible accessibility barriers. 

Two large accessibility evaluation consultant companies, Deque and Siteimprove, have their own paid semi-automated accessibility evaluation tools. \textcite{dequecoverage_semi} extended the same coverage study to semi-automated test tools mentioned in Section \ref{coverage}. They discovered that the coverage of found accessibility barriers is increased by 23\% when using their tool \citep{dequecoverage_semi}. The guided testing is incorporated to their test automation to help evaluators on a more systematic review. For example, in Deque's tool the guided testing to check if an alternative text for an image or page title is descriptive is prompted in a wizard to the evaluator in the following way: \blockquote{The page title is 'Understanding WCAG 2.2 | WAI | W3C'. Does it accurately describe the purpose of the page?}. This question leaves the decision solely up to the evaluator to answer. To answer the question reliably, the evaluator needs to form an understanding of the page content.


\section{Potential of Artificial Intelligence}

With the emergence of multiple Generative AI tools for different context, such as music, image or text generation, the use of AI and machine learning models in web accessibility has sparked discussion and gained focus in the field of accessibility. Companies such as AudioEye and Deque have incorporated AI features in their tools \citep{deque_igt, boia_improve_accessibility}. In addition, the topic of Generative AI use in web accessibility has been discussed by W3C WCAG Working Group \citep{ai_wcag_email}. 

One of the most commonly found issue is a missing descriptive alternative text for images \citep{webaimmillions, dequecoverage}. Therefore, the potential of AI in generating the alternative text for images is an interesting topic with multiple viewpoints \citep{ai_wcag_email, boia_alt_text, potential_for_ai}. However, even though there are tools to generate descriptions to images, the descriptions are not always considered appropriate alternative text for image as they do not take into account context around the image \citep{accessibility_and_ai, boia_alt_text}. Nevertheless, within the accessibility community, some say that AI should work as an assistant that should not make automatic decisions regarding accessibility \citep{ai_wcag_email, accessibility_and_ai}.

\textcite{chen2020unblind} conducted a study where they trained a machine learning model to describe labels for image buttons in Android applications named LabelDroid. The results show that when training models for specific context, the outcome can be helpful and even provide better description then android developers with approximately one year of experience in the field. In addition, the potential of AI in accessibility is not solely within the guidelines or conformance reviews, it can also be used to simplify text content or in incorporated into accessibility tools used by people with disabilities.

