\chapter{Accessibility evaluation\label{accessibility_evaluation}}

As web services undergo rapid development and continuous deployment, a continuous accessibility evaluation is crucial, mandated by both legislative organizations and end-users. The primary goal of assessing web accessibility is to promote digital inclusion by identifying and eliminating barriers, thereby expanding access to a wider audience. Multiple accessibility evaluation tools are available to help web content developers in ensuring inclusivity and conformance to accessibility standards.

\section{Tools}

Accessibility evaluation tools (AET) should be used to check if a web page conforms to accessibility guidelines. Accessibility evaluators parses the HTML and CSS code of a web page and checks that the web page conforms to the accessibility standards. These tools are helpful for quick accessibility evaluation. However, human evaluation is always required to check for full conformance as these tools might produce misleading results \citep{wcagevaluationtools}. Therefore, accessibility evaluation tools should be used as assistive evaluators when examining the conformance of a web page.

In a study conducted by \cite{govukaccessibilityresults} they created a page with 142 accessibility issues. This study shows that automatically these tools cover only around 30 \% of these issues \citep{govukaccessibilityresults}. Excluding the Nu HTML Checker, as it is not an evaluation tool, the lowest score was 17 \% by Google ADT and highest scoring tool was SortSite. 

- Deque, axecore developer, 0 false positives, other viewpoint, see \cite{dequecoverage}.
- False positives, false negatives, tool effectiveness (https://link-springer-com.libproxy.helsinki.fi/content/pdf/10.1007/s10209-004-0105-y.pdf)




- https://dl-acm-org.libproxy.helsinki.fi/doi/pdf/10.1145/3563137.3563148
-> Unified reporting does not exist between EU Member states
-> There's no standard on how to evaluate accessibility
-> 

- Monitoring with automated testing tools; https://eur-lex.europa.eu/legal-content/EN/TXT/HTML/?uri=CELEX:32018D1524

\section{Manual / Beyond reach}


\section{Use of AI}

- Talk about LabelDroid
- Mention accessibility overlay tools; https://commission.europa.eu/resources-partners/europa-web-guide/design-content-and-development/accessibility/testing-early-and-regularly/accessibility-overlays_en, might use AI, has been seen to cause more harm.